\documentclass[a4paper, 11pt, landscape]{article}

\usepackage[utf8]{inputenc}
\usepackage[T1]{fontenc}

\usepackage[english,ngerman]{babel}

\usepackage{multicol}
\setlength{\columnseprule}{0.4pt}

\usepackage[left=4.5mm, right=4.5mm, top=4.5mm, bottom=6mm, landscape, nohead, nofoot]{geometry}

\usepackage{amsmath, amssymb}
\newcommand\N{\ensuremath{\mathbb{N}}}
\newcommand\R{\ensuremath{\mathbb{R}}}
\newcommand\Z{\ensuremath{\mathbb{Z}}}
\newcommand\Q{\ensuremath{\mathbb{Q}}}
\newcommand\C{\ensuremath{\mathbb{C}}}
\renewcommand\O{\ensuremath{\emptyset}}

\usepackage{enumitem}
\usepackage{paralist}
\usepackage{xcolor}
\newcommand{\todo}[1]{\textcolor{red}{TODO: #1}\PackageWarning{TODO:}{#1!}}
\usepackage{parskip}
\usepackage{tcolorbox}

\newtcolorbox{satz}[2][]
{
  colframe = red!10,
  colback  = white,
  coltitle = red!10!black,  
  title    = \textbf{#2},
  #1,
}

\newtcolorbox{definition}[2][]
{
  colframe = blue!10,
  colback  = white,
  coltitle = blue!10!black,  
  title    = \textbf{#2},
  #1,
}

\usepackage[small,compact]{titlesec}
% coloured section headings for easier read
\titleformat{name=\section}[block]
  {\sffamily}
  {}
  {0pt}
  {\colorsection}
\newcommand{\colorsection}[1]{%
	\colorbox{red!10}{\parbox[t][0em]{\dimexpr\columnwidth-2\fboxsep}{\thesection\ #1}}}


\titleformat{name=\subsection}[block]
  {\sffamily}
  {}
  {0pt}
  {\subcolorsection}
\newcommand{\subcolorsection}[1]{%
	\colorbox{orange!10}{\parbox[t][0em]{\dimexpr\columnwidth-2\fboxsep}{\thesubsection\ #1}}}


\titleformat{name=\subsubsection}[block]
  {\sffamily}
  {}
  {0pt}
  {\subsubcolorsection}
\newcommand{\subsubcolorsection}[1]{%
	\colorbox{blue!10}{\parbox[t][0em]{\dimexpr\columnwidth-2\fboxsep}{\thesubsubsection\ #1}}}

\begin{document}
\begin{multicols}{3}
    %! TEX root  = ./main.tex

\section{Allgemein}

\subsection{Potenzen und Wurzeln}
\begin{compactitem}
    \item 
        \begin{inparaitem}
            \item $a^1 = a$
            \item $a^0 = 1$
        \end{inparaitem}
    \item $a^{-n} = \frac{1}{a^n}$
    \item 
        \begin{inparaitem}
            \item $a^{\frac{1}{n}} = \sqrt[^n]{a}$
            \item $a^{\frac{m}{n}} = \sqrt[^n]{a^m}$
        \end{inparaitem}
    \item $a^x = e^{x \ln a}$
    \item $a^ma^n = a^{m + n}$
    \item $\frac{a^m}{a^n} = a^{m - n}$
    \item $(a^m)^n = a^{m n}$
    \item $a^nb^n = (ab)^n$
    \item $\frac{a^n}{b^n} = \left( \frac{a}{b} \right)^n$
\end{compactitem}

\subsection{Logarithmen}
$y = \log_a x \iff a^x = x$
\begin{compactitem}
    \item $a^{\log_a x} = x$
    \item $\log_a a^x = x$
    \item 
        \begin{inparaitem}
            \item $\log_a a = 1$
            \item $\log_a 1 = 0$
        \end{inparaitem}
    \item $\log(uv) = \log(u) + \log(v)$
    \item $\log(\frac{u}{v}) = \log(u) - \log(v)$
    \item $\log(u^r) = r \log(u)$
    \item $\log_a x = \frac{\log_b x}{\log_b a}$
\end{compactitem}

\subsection{Spezielle Summen}
\begin{tabular}{l | l | c}
    $\sum_{k=1}^{n} k$ & $1 + 2 + \dots + n$ & $\frac{n(n+1)}{2}$\\\hline
    $\sum_{k=1}^{n} k^2$ & $1 + 4 + \dots + n^2$ & $\frac{n(n+1)(2n+1)}{6}$\\\hline
    $\sum_{k=1}^{n} k^3$ & $1 + 8 + \dots n^3$ & $\left( \sum_{k=1}^{\infty} k  \right)^2$\\\hline
    $\sum_{k=1}^{n} (2k - 1)$ & $1 + 3 + \dots + (2n -1)$ & $n^2$\\\hline
    $\sum_{k=1}^{n} (2k - 1)^2$ & $1 + \dots + (2n - 1)^2$ & $\frac{n(2n-1)(2n+1)}{3}$\\\hline
    $\sum_{k=0}^{n-1} q^k$ & $1 + q + \dots = \frac{q^n-1}{q-1}$ & $\frac{1 - q^n}{1-q}, \ q \not\in \{0, 1\}$\\\hline
\end{tabular}

\subsection{Trigonometrische Funktionen}
\begin{compactdesc}
    \item[Sinussatz:] $\frac{a}{\sin \alpha} = \frac{b}{\sin \beta} = \frac{c}{\sin \gamma} = 2r$
    \item[Cosinussatz:] $a^2 = b^2 + c^2 - 2ab \cos \alpha$
\end{compactdesc}
\begin{compactitem}
    \item $\sin^2a + \cos^2a = 1$
    \item $\tan a = \frac{\sin a}{\cos a}$
    \item $1 + \tan^2 a = \frac{1}{\cos^2 a}$
\end{compactitem}

\subsubsection{Hyperbolisch}
\begin{compactitem}
    \item $\cosh^2(x) - \sinh^2(x) = 1$
    \item $\sinh(a + b) = \sinh(a)\cosh(b) + \cosh(a)\sinh(b)$
    \item $\cosh(a + b) = \cosh(a)\cosh(b) + \sinh(a)\sinh(b)$
\end{compactitem}

\subsubsection{Winkel}
\begin{tabular}{c c c c || c c c c}
    deg   & rad                & $\sin$                 & $\cos$                  & deg & rad & $\sin$ & $\cos$\\\hline
    $0$   & $0$                & $0$                    & $1$                     &
    $30$  & $\frac{\pi}{6}$    & $\frac{1}{2}$          & $\frac{\sqrt{3}}{2}$\\
    $45$  & $\frac{\pi}{4}$    & $\frac{\sqrt{2}}{2}$   & $\frac{\sqrt{2}}{2}$    &
    $60$  & $\frac{\pi}{3}$    & $\frac{\sqrt{3}}{2}$   & $\frac{1}{2}$\\
    $90$  & $\frac{\pi}{2}$    & $1$                    & $0$                     &
    $120$ & $\frac{2\pi}{3}$   & $\frac{\sqrt{3}}{2} $  & $\frac{-1}{2}$\\
    $135$ & $\frac{3 \pi}{4}$  & $\frac{\sqrt{2}}{2}$   & $\frac{-\sqrt{2}}{2}$   &
    $150$ & $\frac{5 \pi}{6}$  & $\frac{1}{2}$          & $\frac{-\sqrt{3}}{2}$\\
    $180$ & $\pi$              & $0$                    & $-1$                    &
    $210$ & $\frac{7\pi}{6}$   & $\frac{-1}{2}$         & $\frac{-\sqrt{3}}{2}$\\
    $225$ & $\frac{5\pi}{4}$   & $\frac{-\sqrt{2}}{2}$  & $\frac{-\sqrt{2}}{2}$   &
    $240$ & $\frac{4\pi}{3}$   & $\frac{-\sqrt{3}}{2}$  & $\frac{-1}{2}$\\
    $270$ & $\frac{3 \pi}{2}$  & $-1$                   & $0$                     &
    $300$ & $\frac{5 \pi}{3}$  & $\frac{-\sqrt{3}}{2} $ & $\frac{1}{2}$\\
    $315$ & $\frac{7 \pi}{4}$  & $\frac{-\sqrt{2}}{2} $ & $\frac{\sqrt{2}}{2}$    &
    $330$ & $\frac{11 \pi}{6}$ & $\frac{-1}{2} $        & $\frac{\sqrt{3}}{2}$\\
\end{tabular}

\subsubsection{Reduktion}
\resizebox{\columnwidth}{!}{\begin{tabular}{l | l | l}
    $\sin \frac{\pi}{2} - a = \cos a$ & $\cos \frac{\pi}{2} - a = \sin a$ & $\tan \frac{\pi}{2} - a = \frac{1}{\tan a}$\\
    $\sin \frac{\pi}{2} + a = \cos a$ & $\cos \frac{\pi}{2} + a = -\sin a$ & $\tan \frac{\pi}{2} + a = \frac{-1}{\tan a}$\\
    $\sin \pi - a = \sin a$ & $\cos \pi - a = -\cos a$ & $\tan \pi - a = -\tan a$\\
    $\sin \pi + a = -\sin a$ & $\cos \pi + a = -\cos a$ & $\tan \pi + a = \tan a$\\
    $\sin 2\pi -a = -\sin a$ & $\cos 2\pi - a = \cos a$ & $\tan 2 \pi - a = -\tan a$\\
    $\sin -a = -\sin a$ & $\cos - a = \cos a$ & $\tan - a = -\tan a$\\
\end{tabular}}


    %! TEX root  = ./main.tex

\section{Zahlen und Mengen}

\subsection{Körper}
\begin{compactdesc}
    \item[Addition:] $\R \times \R \overset{+}{\to} \R, \quad (x, y) \mapsto x + y$
        \begin{compactenum}[{A}.1]
            \item \textbf{Assoziativität:} $x + \left( y + z \right) = \left( x + y \right) + z  \quad \forall x,y,z \in \R$
            \item \textbf{Neurales Element:} $x + 0 = x \quad \forall x \in \R$
            \item \textbf{Inverse Element:} $x + y = 0 \quad  \forall x \in \R \quad \exists y \in \R$
            \item \textbf{Kommunikativität:} $x + y = y + x \quad \forall x,y \in  \R$
        \end{compactenum}
     Werden alle Axiome erfüllt $\implies \R$ ist abelisch.
 \item[Multiplikation:] $\R \times \R \overset{\cdot}{\to} \R, \quad (x, y) \mapsto x \cdot y$
        \begin{compactenum}[{M}.1]
            \item \textbf{Assoziativität:} $x \cdot \left( y \cdot z \right) = \left( x \cdot y \right) \cdot z \quad \forall x,y,z \in \R$
            \item \textbf{Neurales Element:} $x \cdot  1 = x \quad \forall x \in \R$
            \item \textbf{Inverses Element:} $x \cdot y = 1 \quad \forall  x \in \R \quad \exists y \in \R$
            \item \textbf{Kommunikativität:} $x \cdot y = y \cdot x \quad \forall x,y \in \R$
        \end{compactenum}
    Werden alle Axiome erfüllt $\implies R^* = \R \setminus 0$ ist abelisch.
    \item[Distributivität:] Macht \textbf{Addition} und \textbf{Multiplikation} verträglich
        \begin{compactenum}[{D}.1]
            \item \textbf{Distributivität:} $x \cdot \left( y + z \right) = x \cdot y + x \cdot z \quad \forall x,y,z \in \R$
        \end{compactenum}
    \item[Ordnung ($\le $):] gibt es auf $\R$
        \begin{compactenum}[{O}.1]
        \item \textbf{Reflexivität:} $x \le x \quad \forall x \in \R$
        \item \textbf{Transitivität:} $x \le y \wedge y \le z \implies x \le z$
        \item \textbf{Antisymmetrie/Identität:} $x \le y \wedge y \le x \implies x = y$
        \item \textbf{Total:} $x \le y \vee y \le x \quad \forall x,y \in \R$
        \end{compactenum}
    \item[Kompatibilität:] Ordnung ist mit Körperaxiome kompatibel
        \begin{compactenum}[{K}.1]
            \item $x \le y \implies x + z \le y + z \quad \forall x,y,z \in \R$
            \item $x \cdot  y \ge 0 \quad \forall x \ge 0, y \ge 0$
        \end{compactenum}
    \item[Ordnungsvollständigkeit ($V$):] Existiert für $\R$
        \begin{compactenum}
            \item $A, B \subset \R, \quad A, B \neq \O$
            \item $a \le  b \quad \forall  a \in  A,b \in B$
        \end{compactenum}
        $\implies \exists c \in  \R, \forall a \in  A, b \in  B, \quad a \le c \wedge c \le b$
\end{compactdesc}

\todo{Axioms}
\begin{compactenum}
    \item $\forall x \in \R: (-1) \cdot x = -x$  
    \item $(-1) \cdot  (-1) = 1$
    \item $\forall x \in  \R: x^2 \ge 0$
    \item $0 < 1 < 2 < \ldots$
    \item $\forall x > 0: \frac{1}{x} > 0$
    \item $item$
\end{compactenum}

\subsection{Archimedisches Prinzip}
\begin{compactenum} \item $\forall x,y \in \R, x > 0, \exists n \in \N: y \le n \cdot x$ 
    \item $\forall x \in \R, \exists n \in \Z: n \le x < n + 1$
\end{compactenum}

\subsection{Min, Max, Abs}
\begin{compactdesc}
    \item[Max:] $\max \{x,y\} = \max(x, y) \ \forall x,y \in \R$
    \item[Min:] $\min \{x,y\} = \min(x, y) \ \forall x,y \in \R$
    \item[Absolutbetrag:] $\forall x \in \R: |x| = \max \{-x, x\} $
    \begin{compactenum}
        \item $|x| \ge 0 \ \forall x \in \R$
        \item $|x \cdot y| = |x| \cdot |y| \ \forall x \in \R$
        \item $|x + y| \le |x| + |y| \ \forall x \in \R$
        \item $|x + y| \ge ||x| - |y|| \ \forall x \in \R$
    \end{compactenum} 
\end{compactdesc}

\subsection{Yung'sche Ungleichung}
$\forall x,y \in \R \ \forall \epsilon > 0: \quad 2|x \cdot  y| \le \epsilon \cdot  x^2 + \frac{1}{\epsilon} \cdot y^2$

\subsection{Intervall}
Teilmenge von $\R$
\begin{compactenum}
    \item $\forall a,b \in R, a \le b$
        \begin{compactenum}
            \item $[a,b] = [a, b] = \{x \in \R | a \le  x \le b\} $
            \item $[a,b) = [a, b[ = \{x \in \R | a \le  x < b\} $ 
            \item $(a,b] = ]a, b] = \{x \in \R | a <  x \le b\} $
            \item $(a,b) = ]a, b[ = \{x \in \R | a <  x < b\} $
        \end{compactenum}
    \item $\forall  a \in \R$
        \begin{compactenum}
            \item $[a, +\infty[ = \{x \in \R | x \geq a\} $
            \item $]a, +\infty[ = \{x \in \R | x > a\} $
            \item $]-\infty, a] = \{x \in \R | x \le a\} $
            \item $]-\infty, a[ = \{x \in \R | x < a\} $
        \end{compactenum}
    \item $\R = ]-\infty, +\infty[$
\end{compactenum}

\subsection{Schranken}
\begin{compactdesc}
    \item[Teilmenge:] $A \subset \R, A \neq \O$
    \item[Obere Schranke:] $c \in R$ von $A$ falls  $\forall a \in A: a \le c$
    \item[Nach Oben Beschränkt:] $A$ hat eine obere Schranke
    \item[Untere Schranke:] $c \in \R$ von $A$ falls $\forall a \in A: a \ge c$
    \item[Nach Unten Beschränkt:] $A$ hat eine untere Schranke
    \item[Supremum:] $s:= \sup A$ ist kleinste obere Schranke von $A$
    \item $\underbrace{\left( \forall a \in  A | a \ge  s \right)}_{s \text{ ist eine obere Schranke von } A} \wedge \underbrace{\left( \forall  \epsilon > 0 \ \exists a \in A | a > s - \epsilon \right)}_{s \text{ ist die kleinste obere Schranke von } A}  $
    \item[Infimum] $l:= \inf A$ ist grösste untere Schranke von $A$
    \item $\underbrace{\left( \forall a \in  A | a \ge  l \right)}_{l \text{ ist eine untere Schranke von } A} \wedge \underbrace{\left( \forall  \epsilon > 0 \ \exists a \in A | a > l + \epsilon \right)}_{l \text{ ist die grösste untere Schranke von } A}  $
\end{compactdesc}

\begin{compactitem}
    \item $a \le b \ \forall a \in A, b \in B \implies \sup A \le \inf B$
    \item $A, B \subseteq \R, c \in \R$
        \begin{compactitem}
            \item $c \cdot A := \{c \cdot  a | a \in  A\} $
            \item $A + B := \{a + b | a \in A, b \in B\} $
        \end{compactitem}
    \item $\sup \{A \cup B\} = \max \{\sup A, \sup B\}  $
    \item $\sup \left( A + B \right) = \sup A + \sup B $
    \item $\sup \left( c \cdot A \right) = \begin{cases}
        c \cdot \sup A & \text{if } c > 0\\
        c \cdot \inf A & \text{if } c < 0
    \end{cases} $
\end{compactitem}

\begin{compactenum}
    \item $A \subset \R$ nach oben beschränkt $\implies$ Menge der oberen Schranken ist  im Intervall $[\sup A, +\infty[$
    \item $A \subset \R$ nach unten beschränkt $\implies$ Menge der unteren Schranken ist  im Intervall $]-\infty, \inf A]$
\end{compactenum}

$A \subset  B \subset  \R$:
\begin{compactenum}
    \item $B$ nach oben beschränkt $\implies \sup A \le \sup B$
    \item $B$ nach unten beschränkt $\implies \inf A \ge \inf B$
\end{compactenum}

$A \subset \R$:
\begin{compactenum}
    \item $A$ nach oben unbeschränkt $\implies \sup A = +\infty$
    \item $A$ nach unten unbeschränkt $\implies \inf A = -\infty$
\end{compactenum}

\subsection{Kardinalität}
\begin{compactdesc}
    \item[Gleichmächtig:] $X,Y \subset \R \ \exists$ Bijektion $f : X \to Y$
    \item[Endlich:] falls $X = \O$ oder $\exists n \in \N: \{1, 2, \ldots, n\}, X$ gleichmächtig
    \item[Abzählbar:] falls $X$ endlich oder gleichmächtig wie $\N$
\end{compactdesc}

\section{Euklidischer Raum}
\todo{Euklidischer Raum}
\section{Komplexe Zahlen}
\todo{Komplexe Zahlen}

    %! TEX root = ./main.tex

\section{Folgen}

\begin{compactdesc}
    \item[Folge:] $\left( a_n \right)_{n \ge a  > 0}$ ist Funktion $a:\N^* \to \mathbb{A}, n \mapsto a_n, \mathbb{A}$ ist beliebiges Set.
    \item[Konvergent:] ist $(a_n)_{n \in \N}$ falls $\exists l \in R: \quad \forall \epsilon > 0$ das Set $\left\{ n \in \N | a_n \not\in  ] l - \epsilon, l + \epsilon[ \right\} = \left\{ n \in N^* | \left| a_n - l \right| \ge \epsilon  \right\} $ endlich ist
        \begin{compactitem}
            \item $\forall  \epsilon > 0 \ \exists  N \ge 1: \quad |a_n - l| < \epsilon \ \forall n \ge N$
            \item $(a_n)_{n \in \N}$ konvergente $\implies (a_n)_{n \in \N}$ beschränkt.
            \item Es gibt $2$ Arten von divergenten Folgen
        \end{compactitem}
    \item[Grenzwert:] $(a_n)_{n \in \N}$ konvergiert gegen $a \iff \lim_{n \to \infty} a_n = a \iff \forall \epsilon > 0, \exists n_0 \in N, \forall n \ge n_0: \left| a_n - a \right| < \epsilon$
        \begin{compactitem}
            \item $\lim_{n \to \infty} a_n = \lim_{n \to \infty} a_{n+k} \forall k \in \N $
        \end{compactitem}
\end{compactdesc}

\subsection{Rechenregeln}
$\forall (a_n)_{n \in \N}, (b_n)_{n \in \N}$ konvergent, mit $\lim_{n \to \infty} a_n = a, \lim_{n \to \infty} b_n = b$
\begin{compactenum}
    \item $(a_n + b_n)_{n \in \N}$ ist konvergent mit $\lim_{n \to \infty} (a_n + b_n) = a + b$
    \item $(a_n \cdot  b_n)_{n \in \N}$ ist konvergent mit $\lim_{n \to \infty} (a_n \cdot  b_n) = a \cdot b$
    \item $\left( \frac{a_n}{b_n} \right) _{n \in \N}$ ist konvergent mit $\lim_{n \to \infty} \left( \frac{a_n}{b_n} \right) = \left( \frac{a}{b} \right)$ if $b_n \neq 0 \ \forall n \in N \wedge  b \neq 0$
    \item Falls $\exists K \ge 1, a_n \le  b_n \ \forall n \ge K \implies a \le b$
    \item $\lim_{n \to \infty} \sqrt{a_n} = \sqrt{\lim_{n \to \infty} a_n}$
\end{compactenum}

\subsection{Monotonie}
\begin{compactdesc}
    \item[Monoton Wachsend:] $a_n \le a_{n+1} \quad \forall n \in \N$
    \item[Strikt Monoton Wachsend:] $a_n < a_{n+1} \quad \forall n \in \N$
    \item[Monoton Fallend:] $a_n \ge a_{n+1} \quad \forall n \in \N$
    \item[Strikt Monoton Fallend:] $a_n > a_{n+1} \quad \forall n \in \N$
\end{compactdesc}

\subsection{Einschliessungskriterium}
    $\lim_{n \to \infty} a_n = \lim_{n \to \infty} b_n = \alpha \in \R \ \exists K \in \N \ \exists \left( c_n \right)_{n \in \N}: a_n \le  c_n \le b_n \ \forall n \ge K \implies \lim_{n \to \infty} c_n = \alpha$

\subsection{Weierstrass / Monoton Konvergenz Satz}
\begin{compactitem}
   \item $(a_n)_{n \in \N}$ monoton wachend und nach oben beschränkt $\implies (a_n)_{n \in \N}$ konvergiert und $\lim_{n \to \infty} a_n = \sup \left\{ a_n : n \ge 1 \right\}$
   \item $(a_n)_{n \in \N}$ monoton fallend und nach unten beschränkt $\implies (a_n)_{n \in \N}$ konvergiert und $\lim_{n \to \infty} a_n = \inf \left\{ a_n : n \ge 1 \right\}$
\end{compactitem}

\subsection{Funktionen und deren Grenzwert}
\begin{tabular}{l | c | c}
    Funktion & Grenzwert & Bedingung\\\hline
    $\lim_{n \to \infty} a^n$ & $0$ & $|a| < 1$\\\hline
    $\lim_{n \to \infty} \sqrt[^n]{a}$ & $1$ & $a > 0$\\\hline
    $\lim_{n \to \infty} \sqrt[^n]{n^a}$ & $1$ & $a > 0$\\\hline
    $\lim_{n \to \infty} \sqrt[^n]{n}$ & $1$\\\hline
    $\lim_{n \to \infty} \frac{\log_an}{n}$ & $0$ & $a > 1$\\\hline
    $\lim_{n \to \infty} \frac{n^k}{a^n}$ & $0$ & $a > 1$\\\hline
    $\lim_{n \to \infty} \frac{a^n}{n!}$ & $0$\\\hline
    $\lim_{n \to \infty} \sum_{k=1}^{n} \frac{1}{k}$ & $\infty$\\\hline
    $\lim_{n \to \infty} \left( 1 + \frac{1}{n} \right)^n$ & $e$\\\hline
    $\lim_{n \to \infty} \left( 1 + \frac{a}{n} \right)^n$ &e$^a$\\\hline
    $\lim_{n \to \infty} \left( 1 - \frac{1}{n} \right)^n$ & $\frac{1}{e}$\\\hline
    $\lim_{n \to 0} \frac{\sin n}{n}$ & $1$\\\hline
    $\lim_{n \to 1} \frac{\ln n}{n - 1}$ & $1$\\\hline
    $\lim_{n \to \infty} \frac{n^m}{\exp(an)}$ & $0$ & $m \in \R, a > 0$\\\hline
    $\lim_{n \to 0} \frac{\exp(n) - 1}{n}$ & $1$\\\hline
    $\lim_{n \to 0} \frac{\ln(1 + n)}{n}$ & $1$\\\hline
    $\lim_{n \to \infty} \frac{\ln n}{n^a}$ & $0$ & $a > 0$\\\hline
    $\lim_{n \to 0} \frac{a^n - 1}{n}$ & $\ln a$ & $a > 0$\\\hline
    $\lim_{n \to 0} (n^a \ln n)$ & $0$ & $a > 0$\\\hline
\end{tabular}

\subsection{Bernoulli Ungleichung}
$\left( 1 + x \right)^{n} \ge  1 + n \cdot x \ \forall n \in N, x > -1 $

\subsection{Limes Superior und Limes Inferior}
Jede beschränkte Folge $(a_n)_{n \in \N}$ kann in zwei monotone Folgen $(b_n)_{n \in \N}$ und $(c_n)_{n \in \N}$ geteilt werden.
\begin{compactenum}
    \item $\forall n \ge 1: b_n = \inf \{a_k | k \ge n\}$ und $c_n = \sup \{a_k | k \ge n\} $
    \item $b_n \le b_{n+1}$ und $c_n \ge c_{n+1} \quad \forall n \in \N$
    \item $(b_n)_{n \in \N}$ monoton wachsend, $(c_n)_{n \in \N}$ monoton fallend
    \item $b_n$ und $c_n$ sind beschränkt $\implies$ konvergent
    \item \textbf{Limes Inferior:} $\liminf_{n \to \infty} a_n := \lim_{n \to \infty} b_n $
    \item \textbf{Limes Superior:} $\limsup_{n \to \infty} a_n := \lim_{n \to \infty} c_n $
    \item $b_n \le  c_n \implies \liminf_{n \to \infty} a_n \le  \limsup_{n \to \infty} a_n $
\end{compactenum}

\begin{compactitem}
    \item $(a_n)_{n \in \N}$ konvergiert $\iff \ (a_n)_{n \in \N}$ beschränkt und $\liminf_{n \to \infty} a_n = \limsup_{n \to \infty} a_n$
\end{compactitem}

\subsection{Cauchy Kriterium}
\begin{compactdesc}
    \item[Cauchy-Folge:] $(a_n)_{n \in \N}$ falls $\forall \epsilon > 0 \ \exists N \in \N: \quad \forall m,n \ge N \ |a_n - a_m| < \epsilon$
        \begin{compactitem}
            \item Abstand zwischen Folgegliedern wird mit wachsendem Index beliebig klein
        \end{compactitem}
\end{compactdesc}

\begin{compactitem}
    \item $a_n$ Cauchy-Folge $\implies a_n$ beschränkt
    \item $a_n$ konvergent $\iff a_n$ Cauchy-Folge
    \item $(a_n)_{n \in \N}$ konvergiert $\iff \ \forall \epsilon > 0 \ \exists \N \ge 1: |a_n - a_m| < \epsilon \quad \forall n,m \ge \N$
    \item $a_n$ nicht Cauchy-Folge $\implies a_n$ divergent
\end{compactitem}

\subsection{Abgeschlossener Teilintervall}
Teilmenge $I \subset \R$
\begin{compactenum}
    \item $[a, b] \quad a \le b, a,b \in \R \implies \text{L}(I) = b - a$
    \item $[a, +\infty[ \quad a \in \R \implies \text{L}(I) = \infty$
    \item $]-\infty, a] \quad a \in \R \implies \text{L}(I) = \infty$
    \item $]-\infty, +\infty[ = \R \implies \text{L}(I) = \infty$
\end{compactenum}

\begin{compactitem}
    \item $I \subset \R$ beschränkt$ \ \iff \text{L}(I) < + \infty$
    \item $I \subset \R$ ist abgeschlossen $\iff $ für jede konvergierende $(a_n)_{n \in \N}$ aus Elementen in $I$ muss $\lim_{n \to \infty} a_n \in I$
    \item $I=[a, b], J=[c, d], a \le b, c \le d, \ a,b,c,d \in \R$, falls $c \le a$ und $b \le d \implies I \subset J$
        \begin{compactitem}
            \item $\text{L}(I) = b - a \le  d - c = \text{L}(J)$
        \end{compactitem}
    \item Monoton fallende Folge von Teilmengen von $\R$ ist eine Folge $(X_n)_{n \in \N}, X_n \subset \R$ mit $X_1 \supseteq X_2, \supseteq \dots \supseteq X_n \supseteq \dots $
\end{compactitem}

\subsection{Cauchy-Cantor}
Für absteigende Folge geschlossener Intervalle $I_1 \supseteq \dots  \supseteq I_n \supseteq \dots $ mit $\text{L}(I_1) < + \infty$ gilt $\bigcap_{n \ge 1} I_n \neq \O $. Falls $\lim_{n \to \infty} \text{L}(I_n) = 0 \implies \bigcap_{n \ge  1} I_n =\{x\} \ x \in \R $.

\subsection{Teilfolge}
Teilfolge von $(a_n)_{n \in \N}$ ist $(b_n)_{n \in \N}$ wobei $b_n = a_{\text{l}(n)}$ für $l:\N^* \to \N^*$ mit der Eigenschaft $\text{l}(n) < l(n + 1) \forall n \ge 1$
\begin{compactitem}
    \item Entsteht durch weglassen von Folgengliedern.
    \item $(a_n)_{n \in \N}$ konvergent $\implies (a_{\text{l}(n)})_{n \in \N}$ konvergent für alle Teilfolgen.
\end{compactitem}

\subsection{Bolzano-Weierstrass}
Jede beschränkte Folge besitzt eine konvergente Teilfolge.
\begin{compactitem}
    \item $(a_n)_{n \in \N}$ beschränkt $\implies$ für jede beschränkte Teilfolge $(b_n)_{n \in \N}$ gilt $\liminf_{n \to \infty} a_n \le \lim_{n \to \infty} b_n \le  \limsup_{n \to \infty} a_n$
    \item Es gibt je eine Teilfolgen von $(a_n)_{n \in \N}$ die $\liminf_{n \to \infty} a_n$ resp. $\limsup_{n \to \infty} a_n$ als Limes annehmen.
\end{compactitem}

\subsection{Rezept: Konvergenz und Grenzwert}
\begin{compactdesc}
    \item[Geschlossene Formel:]
        \begin{inparaitem}
            \item auf Bruch erweitern
            \item $n$ ausklammern und kürzen
            \item $n$ in Nenner bekommen
        \end{inparaitem}
    \item[Rekursive Definition:]
        \begin{inparaitem}
            \item Geschlossene Formel finden
            \item
                \begin{inparaenum}
                    \item Monotonie zeigen
                    \item Beschränktheit zeigen
                    \item Monoton Konvergenzsatz
                    \item Induktionstrick:
                        \begin{inparaitem}
                            \item $c = \lim_{n \to \infty} a_n = \lim_{n \to \infty} a_{n+1}$
                            \item Solve $c$
                        \end{inparaitem}
                \end{inparaenum}
        \end{inparaitem}
\end{compactdesc}

\section{Reihen}
\begin{compactdesc}
    \item[Folge der Partialsummen:] $(S_n)_{n \in \N} := a_1 + a_2 + \dots + a_n = \sum_{k=1}^{n} a_k$ einer Folge $(a_n)_{n \in \N}$.
        \begin{compactitem}
        \item $(S_n)_{n \in \N}$ konv. $\implies \sum_{k=1}^{\infty} a_k$  konv.
            \item $(S_n)_{n \in \N} \text{ nach oben beschränkt } \iff \sum_{k=1}^{\infty} a_k, a_k \ge 0 \ \forall k \in \N^*$ konvergiert.
            \item Ist monoton steigend.
        \end{compactitem}
    \item[Reihe:] Unendliche Summe $\sum_{k=1}^{\infty} a_k$ einer Folge $(a_n)_{n \in \N}$.
        \begin{compactitem}
            \item Für divergierende Reihen ist die Summe ein Symbol nicht konvergente Folge $(s_n)_{n \in \N}$.
            \item Für konvergente Reihe ist die Summe ein Symbol für den Grenzwert der Folge $(s_n)_{n \in \N}$.
        \end{compactitem}
    \item[Konvergent:] ist $\sum_{k=1}^{\infty} a_k$ falls die Folge $(S_n)_{n \in \N}$ von $(a_n)_{n \in \N}$ konvergiert.
        \begin{compactitem}
            \item $\sum_{k=1}^{\infty} a_k$ konvergiert $\implies \lim_{k \to \infty} a_k = 0 $.
            \item $\lim_{k \to \infty} |a_k| \neq  0 \implies \sum_{k=1}^{\infty} a_k$ divergent.
        \end{compactitem}
    \item[Grenzwert:] $\sum_{k=1}^{\infty} a_k := \lim_{n \to \infty} S_n = \lim_{k \to \infty} \sum_{n=1}^{k} S_n$.
    \item Weglassen von Anfangsgliedern verändert die Konvergenz nicht, verändert ggf. jedoch den Grenzwert.
\end{compactdesc}

\subsection{Bekannte Reihen}
\resizebox{\columnwidth}{!}{\begin{tabular}{l | l | c | c | c}
    \multicolumn{2}{l |}{Reihe}                                      & Wert                                   & konv.               & div.\\\hline
    \multicolumn{2}{l |}{Geometrische Reihe}                         & $q \in \C$                             &                     & \\\hline
    $\sum_{k=0}^{\infty} aq^k$                                       & $a + aq +$                             & $\frac{a}{1-q}$     & $|g| < 1$    & $|q| \ge 1$\\\hline
    $\sum_{k=0}^{\infty} (k+1)q^k$                                   & $1 + 2q +$                             & $\frac{1}{(1-q)^2}$ &              & \\\hline
    \multicolumn{2}{l |}{Harmonische Reihe}                          &                                        &                     & \\\hline
    $\sum_{k=1}^{\infty} \frac{1}{k}$                                &                                        & $\infty$            &              & \\\hline
    $\sum_{k=1}^{\infty} \frac{1}{k^2}$                              &                                        & $\frac{\pi^2}{6}$   &              & \\\hline
    $\sum_{k=1}^{\infty} \frac{1}{k^4}$                              &                                        & $\frac{\pi^4}{90}$  &              & \\\hline
    $\sum_{k=1}^{\infty} \frac{1}{k^a}$                              &                                        &                     & $a > 1$      & $a \le 1$\\\hline
    \multicolumn{2}{l |}{Alternierende Harmon. Reihe}                &                                        &                     & \\\hline
    $\sum_{k=1}^{\infty} \frac{(-1)^{k+1}}{k}$                       &                                        & $\ln 2$             &              & \\\hline
    $\sum_{k=1}^{\infty} \frac{(-1)^{k+1}}{k^2}$                     &                                        & $\frac{\pi^2}{12}$  &              & \\\hline
    $\sum_{k=1}^{\infty} \frac{(-1)^{k+1}}{k^4}$                     &                                        & $\frac{\pi^4}{720}$ &              & \\\hline
    $\sum_{k=0}^{\infty} \frac{(-1)^k}{2k + 1}$                      & $1 - \frac{1}{3} + \frac{1}{5} -$      & $\frac{\pi}{4}$     &              & \\\hline
    \multicolumn{1}{l |}{Teleskopreihe}                              &                                        &                     & \\\hline
    $\sum_{k=1}^{\infty} \frac{1}{k(k+1)}$                           &                                        & $1$                 &              & \\\hline
    \multicolumn{3}{l |}{Exponentialfunktion $z \in \C,$ konv. abs.} &                                        & \\\hline
    $\sum_{k=0}^{\infty} \frac{z^k}{k!}$                             & $1 + z + \frac{z^2}{2!} +$             & $\exp{z}$           &              & \\\hline
    $\sum_{k=0}^{\infty} \frac{(-a)^k}{k!}$                          &                                        & $\frac{1}{e^a}$       &              & \\\hline
    $\sum_{k=0}^{\infty} (-1)^k\frac{x^{2k+1}}{(2k+1)!}$             & $x -\frac{x^3}{3!} + \frac{x^5}{5!} -$ & $ \sin x$           &              & \\\hline
    $\sum_{k=0}^{\infty} (-1)^k\frac{x^{2k}}{(2k)!}$                 & $1-\frac{x^2}{2}+\frac{x^4}{4!}-$      & $\cos x$            &              & \\\hline
    $\sum_{k=0}^{\infty} \frac{x^{2k+1}}{(2k+1)!}$                   & $x+\frac{x^3}{3!}+\frac{x^5}{5!}+$     & $\sinh x$           &              & \\\hline
    $\sum_{k=0}^{\infty} \frac{x^{2k}}{(2k)!}$                       & $1+\frac{x^2}{2}+\frac{x^4}{4!}+$      & $\cosh x$           &              & \\\hline
\end{tabular}}

\subsection{Rechenregeln}
$\forall \sum_{k=1}^{\infty} a_k, \sum_{k=1}^{\infty} b_k$ konvergent, $\alpha \in \C$
\begin{compactenum}
    \item $\sum_{k=1}^{\infty} (a_k + b_k) = \left( \sum_{k=1}^{\infty} a_k \right) + \left( \sum_{k=1}^{\infty} b_k \right)$ konvergent.
    \item $\sum_{k=1}^{\infty} \alpha \cdot a_k = \alpha \cdot \sum_{k=1}^{\infty} a_k$ konvergent.
\end{compactenum}

\subsection{Cauchy Kriterium}
$\sum_{k=1}^{\infty} a_k$ konvergent $\iff \forall \epsilon > 0 \ \exists N \ge 1: \ \left| \sum_{k=n}^{m} a_k \right| < \epsilon \ \forall m \ge  n \ge N$.
\begin{compactitem}
    \item $\sum_{k=1}^{\infty} a_k$ konvergent $\iff \lim_{k \to \infty} \left| \sum_{k=n}^{m} a_k \right| = 0 \ m \ge n$.
    \item $\sum_{k=1}^{\infty} a_k$ konvergent $\implies \lim_{k \to \infty} a_k = 0 $.
    \item $\lim_{k \to \infty} a_k \neq  0 \implies \sum_{k=1}^{\infty} a_k$ divergent.
\end{compactitem}

\subsection{Vergleichssatz}
Für $\sum_{k=1}^{\infty} a_k$ $\sum_{k=1}^{\infty} b_k, 0 \le a_k \le b_k \ \forall k \ge K$:
\begin{compactdesc}
\item[Majoranten Kriterium:] $\sum_{k=1}^{\infty} b_k$ konvergiert $\implies \sum_{k=1}^{\infty} a_k$ konvergiert (konv. abs. falls $|a_k| \le b_k$)
    \item[Minoranten Kriterium:] $\sum_{k=1}^{\infty} k_k$ divergiert $\implies \sum_{k=1}^{\infty} b_k$ divergiert.
\end{compactdesc}

\subsection{Umordnung}
$\sum_{k=1}^{\infty} a'_k$ ist eine Umordnung von $\sum_{k=1}^{\infty} a_k$ falls es eine Bijektion $\phi : \N^* \to \N^*$ gibt so dass $a'_n = a_{\phi(n)}$
\begin{compactitem}
\item Eine nicht subjektive (nur injektiv) Abbildung entspricht der Reihe einer Teilfolge der Folge. $\sum_{n=1}^{\infty} (a_n)_{n \in N}$ konvergent $\implies \sum_{n=1}^{\infty} (a_{\phi(n)})_{n \in \N}$ konvergent.
\end{compactitem}

\subsection{Absolute Konvergenz}
$\sum_{k=1}^{\infty} a_k$ absolut konvergent falls $\sum_{k=1}^{\infty} |a_k|$ konvergent.
\begin{compactitem}
    \item Falls divergent, kann sie nur gegen $+\infty$ divergieren.
    \item $\sum_{k=1}^{\infty}| a_k|$ konvergent $\implies \sum_{k=1}^{\infty} a_k$ konvergent
        \begin{compactitem}
            \item $\left| \sum_{k=1}^{\infty} a_k \right| \le \sum_{k=1}^{\infty} \left| a_k \right|$.
        \end{compactitem}
    \item $a_n \ge 0 \implies$ absolute Konvergenz ist äquivalent zu konvergent.
\end{compactitem}
\begin{compactdesc}
    \item[Bedingt Konvergent:] $\sum_{k=1}^{\infty} a_k$ ist konvergent aber nicht absolut konvergent.
\end{compactdesc}

\subsubsection{Dirichlet}
$\sum_{k=1}^{\infty} a_k$ absolute konvergent $\implies$ jede Umordnung ist konvergent mit demselben Grenzwert.

\subsubsection{Riemann}
$\sum_{k=1}^{\infty} a_k$ konvergent aber nicht absolut konvergent $\implies \exists $ Umordnung $\forall A \in \R \cup \{\pm \infty\}: \sum_{k=1}^{\infty} a_{\phi(n)} = A $.

\subsection{Leibniz}
$(a_n)_{n \in \N}$ monoton fallend, $a_n \ge 0 \ \forall n \in \N$ und $\lim_{n \to \infty} a_n = 0 \implies S:= \sum_{k=1}^{\infty} \left( -1 \right)^{k+1} a_k$ konvergiert.
\begin{compactitem}
    \item $a_1 - a_2 \le S \le a_1$
    \item $(s_n)_{n \in \N}$ beschränkt, $\lim_{n \to \infty} s_{2n} = \lim_{n \to \infty} s_{2n + 1} = s \implies \lim_{n \to \infty} s_n = s$.
\end{compactitem}

\subsection{Quotientenkriterium}
Für $(a_n)_{n \in \N}, a_n \neq 0 \ \forall n \ge 1$:
\begin{compactenum}
    \item $\limsup_{k \to \infty} \frac{\left| a_{n+1} \right|}{\left| a_n \right| } < 1 \implies \sum_{k=1}^{\infty} a_k$ konv. abs.
    \item $\liminf_{k \to \infty} \frac{\left| a_n + 1 \right|}{\left| a_n \right| } > 1 \implies \sum_{k=1}^{\infty} a_k$ divergiert.
\end{compactenum}
\begin{compactitem}
    \item $\exists \lim_{k \to \infty} \left| \frac{a_n + 1}{a_n} \right| =:  L \implies$
        \[
        \begin{cases}
            \sum_{k=1}^{\infty} a_k \text{ absolut konvergent} & \text{ if } L < 1\\
            \sum_{k=1}^{\infty} a_k \text{ konvergent} & \text{ if } L > 1\\
            \text{ versagt Kriterium} & \text{ if } L = 1
        \end{cases}
        \]
    \item Nützlich für $n!, a^n$ und Polynom.
    \item Versagt wenn unendlich viele Glieder $a_n$ der Reihe verschwinden.
\end{compactitem}

\subsection{Wurzelkriterium}
Für $(a_n)_{n \in \N}$:
\begin{compactenum}
    \item $\limsup_{n \to \infty} \sqrt[^n]{\left| a_n \right| } < 1 \implies \sum_{k=1}^{\infty} a_k$ konv. abs.
    \item $\limsup_{n \to \infty} \sqrt[^n]{\left| a_n \right| } > 1 \implies \sum_{k=1}^{\infty} a_k$ und $\sum_{k=1}^{\infty} |a_k|$ divergieren.
\end{compactenum}
\begin{compactitem}
    \item $\exists \lim_{k \to \infty} \sqrt[^n]{ \left| a_n \right| }  = L \implies$
        \[
        \begin{cases}
            \sum_{k=1}^{\infty} a_k \text{ absolut konvergent} & \text{ if } L < 1\\
            \sum_{k=1}^{\infty} a_k \text{und }\sum_{k=1}^{\infty} |a_k| \text{ konvergent} & \text{ if } L > 1\\
            \text{ versagt Kriterium} & \text{ if } L = 1
        \end{cases}
        \]
\end{compactitem}

\subsection{Potenzreihe}
$\text{P}(z) := c_0 + c_1 \cdot z + c_2 \cdot z^2 + \dots = \sum_{k=0}^{\infty} c_k z^k, \ (c_n)_{n \in \N}, z \in \C$.
\begin{compactitem}
    \item Ist absolute konvergent $\forall |z| < p$ und divergiert $\forall |z| > p$.
    \item $
        p:= \begin{cases}
            +\infty & \text{ if } \limsup_{k \to \infty} \sqrt[^k]{\left| c_k \right| }  = 0\\
            \frac{1}{\limsup_{k \to \infty} \sqrt[^k]{|c_k|}} & \text{ if } \limsup_{k \to \infty} \sqrt[^k]{|c_k|} > 0
        \end{cases}
    $
    \item Funktioniert nur wenn $\limsup_{k \to \infty} \sqrt[^k]{|c_k|}$ existiert.
    \item Konvergenzbereich von Potenzreihe ist ein Kreis.
\end{compactitem}
Konvention:
\begin{compactenum}
\item $\{\sqrt[^n]{ \left| a_k \right| } \}$ unbeschränkt $\implies$ wir setzen $p=0 \ (\implies |z| < 0)$.
\item $\{\sqrt[^n]{ \left| a_k \right| } \}$ beschränkt und $\limsup_{k \to \infty} \sqrt[^k]{c_k}=0 \implies$ wir setzen $p=\infty$ ($\implies p(z)$ konvergiert $\forall c \in \C$).
\item $\{\sqrt[^n]{ \left| a_k \right| } \}$ beschränkt und $\limsup_{k \to \infty} \sqrt[^k]{c_k}\neq 0 \implies$ wir setzen $p= \frac{1}{\limsup_{k \to \infty} \sqrt[^k]{c_k}} \ (\implies p(z)$ konvergiert $\forall |z| < p)$.
\end{compactenum}

\subsection{Rezept: Konvergenzradius Berechnen}
\begin{compactenum}
    \item Berechne $|p| = \lim_{n \to \infty} \left| \frac{a_n}{a_{n+1}} \right|$ falls $a_0 \neq 0 \forall n > N$ und Limes definiert oder unendlich ist.
    \item Alternativ verwende $|p| = \frac{1}{\lim_{n \to \infty} \sqrt[^n]{|a_n|}}$
    \item Überprüfe ob $p$ inklusive oder exklusiv ist.
\end{compactenum}

\subsection{Riemann Zeta Funktion}
$\zeta(s) := 1 + \frac{1}{2^s} + \frac{1}{3^s} + \dots = \sum_{n=1}^{\infty} \frac{1}{n^s}, \ s > 0$.
\begin{compactitem}
    \item $\zeta(s), 0 < s \le 1 \implies \zeta(s)$ konvergiert.
    \item $\zeta(s), s > 1 \implies \zeta(s) < \sum_{k=0}^{\infty} \left( \frac{1}{2^{s-1}} \right)^n $ divergent.
\end{compactitem}

\subsection{Doppelfolgen und -reihen}
\begin{compactdesc}
    \item[Doppelfolge:] $(c_{kl})_{k,l \in \N} := a_k \cdot b_l$
    \item[Doppelreihe:] $\sum_{k,l\ge1} c_{kl}$
        \begin{compactitem}
            \item $\sum_{k=0}^{\infty} \sum_{l=0}^{\infty}c_{kl}$ und $\sum_{l=0}^{\infty} \sum_{k=0}^{\infty}c_{kl}$ können mit unterschiedlichem Grenzwert konvergieren.
        \end{compactitem}
    \item[Lineare Anordnung:] von $\sum_{k,l \ge 1} a_{kl}$ ist $\sum_{k=0}^{\infty} b_k$ falls $\exists$ Bijektion $\phi N \to N \times N$ mit $b_k = a_{\phi(k)}$.
\end{compactdesc}

\subsection{Cauchy}
$\exists B \ge 0: \ \sum_{i=0}^{m} \sum_{j=0}^{m} \left| a_{ij} \right| \le B \ \forall m \ge 0 \implies$
\begin{compactitem}
    \item Folgende Reihen konvergieren absolut:
        \begin{compactitem}
            \item $S_i := \sum_{j=0}^{\infty} a_{ij} \ \forall i \ge 0$
            \item $U_j := \sum_{i=0}^{\infty} a_{ij} \ \forall j \ge 0$
            \item $\sum_{i=0}^{\infty} S_i$
            \item $\sum_{j=0}^{\infty} U_j$
        \end{compactitem}
    \item Es gilt $\sum_{i=0}^{\infty} S_i = \sum_{j=0}^{\infty} U_j$
    \item Es konvergiert jede lineare Anordnung der Doppelreihe absolut und hat demselben Grenzwert.
\end{compactitem}

\subsection{Cauchy Produkt}
Produkt von $\sum_{i=1}^{\infty} a_i, \sum_{j=1}^{\infty} b_j$ ist $\sum_{n=0}^{\infty} \left( \sum_{j=0}^{\infty} a_{n-j}b_j \right) = a_0b_0 + (a_0b_1 + a_1b_0) + (a_0b_2 + a_1b_1 + a_2b_0) + \dots$
\begin{compactitem}
    \item Muss nicht immer konvergiere.
    \item Falls $\sum_{i=1}^{\infty} a_i$ und $\sum_{j=1}^{\infty} b_j$ absolut konvergieren $\implies \sum_{n=0}^{\infty} \left( \sum_{j=0}^{\infty} a_{n-j}b_j \right) = \left( \sum_{i=0}^{\infty} a_i \right) \left( \sum_{i=0}^{\infty} b_j \right) $ konvergiert.
\end{compactitem}

\subsection{Folgen Funktionen}
$\forall n$ sein $f_n : \N \to \R$ eine Folge. Wir nehmen an:
\begin{compactenum}
    \item $f(j) := \lim_{n \to \infty} f_n(j)$ existiert $\forall j \in \N$.
    \item $\exists$ Funktion $g: \N \to [0, \infty[$ so dass:
        \begin{compactenum}[{2}.1]
            \item $\left| f_n(j) \right| \le g(j) \ \forall j,n \ge 0$.
            \item $\sum_{j=0}^{\infty} g(j)$ konvergiert.
        \end{compactenum}
\end{compactenum}
dann folgt $\sum_{j=0}^{\infty} f(j) = \lim_{n \to \infty} \sum_{j=0}^{\infty} f_n(j)$.
\begin{compactitem}
\item $\forall z \in \C$ konvergiert die Folge $\left( \left( 1 + \frac{z}{n} \right)^n  \right)_{n \in \N} $ und $\lim_{n \to \infty} \left( 1 + \frac{z}{n} \right)^n = \exp(z)$.
\end{compactitem}

\subsection{Rezept: Konvergenz und Grenzwert}
Gegeben $\sum_{n=1}^{\infty} (a_n)_{n \in N}$
\begin{compactenum}
    \item Ist spezieller Typ $\implies$ betrachte Typ
    \item $\lim_{n \to \infty} a_n \neq 0 \implies$ divergent
    \item Quotientenkriterium anwendbar $\implies$ fertig
    \item Wurzelkriterium anwendbar $\implies$ fertig
    \item $\exists$ konvergente Majoranten $\implies$ konvergent
    \item $\exists$ divergierende Minoranten $\implies$ divergent
    \item Umformen, ausprobieren etc...
\end{compactenum}

    %! TEX root = ./main.tex

\section{Stetige Funktionen}
\subsection{Reellwertige Funktionen}
Für beliebige Menge $D$ ist $\R^D = \{f : D \to \R | f \text{ eine Abbildung}\}$ die Menge alle reellwertigen Funktionen die auf $D$ definiert sind.\\
Addition und skalare Multiplikation bilden mit $\R^D$ einen Vektorraum. Für $f, g \in \R^D, x \in D, \alpha \in \R$:
\begin{compactdesc}
    \item[Addition:] $(f + g)(x) = f(x) + g(x)$
    \item[Skalare Multiplikation:] $\left( \alpha \cdot f \right) (x) = \alpha \cdot f(x)$
    \item[Nullfunktion:] Entspricht dem Nullvektor in $\R^D$ und $\zeta(x) = 0$
    \item[Constate Funkton:]  Entspricht dem Einheitsvektor in $\R^D$ und $\zeta(x) = 1$
    \item[Produkt zweier Funktionen:] $(f \cdot g)(x) := f(x) \cdot g(x)$
    \item[Quotient:] $\frac{f}{g} := D' \to \R, x \mapsto \frac{f(x)}{g(x)}, \ D'= \{x \in D | g(x) \neq 0\} $
    \item[Komposition von Funktionen:] $f: D \to \R$ und $f: E \to \R, f(D) \subset E$ dann $g \circ f: D \to \R, x \mapsto g\left( f(x) \right)$
\end{compactdesc}

\subsection{Beschränktheit}
$f : D \to \R$ ist:
\begin{compactdesc}
    \item[nach oben beschränkt:] falls $f(D) \subset \R$ nach oben beschränkt ist.
    \item[nach unten beschänkt:] falls $f(D) \subset \R$ nach unten beschränkt ist.
    \item[beschränkt:] falls $f(D) \subset \R$ beschränkt ist.
\end{compactdesc}

\subsection{Monotonie}
$f : D \to \R, D \subset R, \ \forall x,y \in D$ ist:
\begin{compactdesc}
    \item[monoton wachsend:] falls $x \le y \implies f(x) \le f(y)$.
    \item[streng monoton wachsend:] falls $x < y \implies f(x) < f(y)$.
    \item[monoton fallend:] falls $x \ge y \implies f(x) \ge f(y)$.
    \item[streng monoton fallend:] falls $x > y \implies f(x) > f(y)$.
    \item[monoton:] falls monoton wachsend oder monoton fallend.
    \item[streng monoton:] falls streng monoton wachsend oder streng monoton fallend.
\end{compactdesc}
\todo{Show monotony}

\subsection{Stetigkeit}
\begin{compactdesc}
    \item[$\mathbf{x_0}$ stetig:] $f: D \to \R$ für $D \subset \R, x_o \in D$  falls  $\forall \epsilon > 0 \ \exists \delta > 0: \left| x - x_0 \right| < \delta \implies \left| f(x) - f_0(x) \right| < \epsilon \ \forall x \in D $.
    \item[stetig:] $f: D \to \R$ falls sie in jedem Punkt von $D$ stetig ist.
\end{compactdesc}

\begin{compactitem}
    \item $f$ ist in $x_0$ stetig $\iff \forall (a_n)_{n \in \N}$ in $D: \ \lim_{n \to \infty} a_n = x_0 \implies \lim_{n \to \infty} f(a_n) = f(x_0)$.
        \begin{compactitem}
            \item $f$ ist in $x_0$ stetig $\iff \lim_{n \to \infty} f(a_n) = f( \lim_{n \to \infty} a_n) \ \forall (a_n)_{n \in \N}$ in $D$.
        \end{compactitem}
\end{compactitem}

\subsection{Rechenregeln}
Für $x_0 \in D \subset \R, \lambda \in \R, f: D \to \R, g: D \to \R$ und $f$ und $g$ stetig in $x_0 \implies$
\begin{compactenum}
    \item $f + g, \lambda \cdot f, f \cdot g$ stetig in $x_0$.
    \item $\frac{f}{g}: D' \to \R, x \mapsto \frac{f(x)}{g(x)}, \ D' = \{x \in D | g(x) \neq 0\}, g(x_0) \neq 0$ ist stetig in $x_0$.
\end{compactenum}
Für $D_1, d_2 \subset \R, f: D_1 \to D_2, g: D_2 \to \R, x_0 \in D_1$. Falls $f$ in $x_0$ und $g$ in $f(x_0)$ stetig $\implies g \circ f : D_1 \to \R$ ist in $x_0$ stetig.
\begin{compactitem}
    \item Falls $f$ auf $D_1$ und $g$ auf $D_2$ stetig $\implies g \circ f$ auf $D_1$ stetig.
\end{compactitem}

\subsection{Polynom}
Funktion $P: \R \to \R, P(x)= a_nx^n + \dots + a_0, a_n, \dots a_0 \in \R$.
\begin{compactdesc}
    \item[Grad:] ist $n$ falls $a_n \neq 0$.
\end{compactdesc}
\begin{compactitem}
    \item Sind auf ganz $\R$ stetig.
    \item $\frac{P}{Q}: R \setminus \{x_1, \dots x_m\}  \to \R, \ x \mapsto \frac{P(x)}{Q(x)}$ ist stetig für $P,Q$ auf $\R$ wobei $ Q \neq 0$  und Nullstellen $x_1, \dots, x_m$ von $Q$.
\end{compactitem}

\subsection{Zwischerwertsatz}
Für Intervall $I \subset R$, stetige Funktion $f: I \to \R$ und $a, b \in I \implies \forall c$ zwischen $f(a)$ und $f(b) \ \exists z$ zwischen $a$ und $b$ mit $f(z) = c$.
\begin{compactitem}
    \item $x, y \in R, x \le y$, $c$ liegt \textbf{zwischen} $x$ und $y$ falls $c \in [x, y]$.
    \item Ein Polynom $P$ mit ungeradem Grad $n$ besitzt mindestens eine Nullstelle in $\R$.
    \item Für $f: [a,b] \to \R$ stetig und $f(a) \cdot f(b) < 0 \implies \exists c \in ]a,b[: f(c) = 0$.
\end{compactitem}

\subsection{Min, Max, Abs}
Für menge $D$ und $f,g: D \to \R$:
\begin{compactdesc}
    \item[Abs:] $|f|(x) := |f(x)|, \ \forall x \in D$
    \item[Max:] $\max(f,g)(x) := \max(f(x), g(x)), \ \forall x \in D$
    \item[Min:] $\min(f,g)(x) := \min(f(x), g(x)), \ \forall x \in D$
\end{compactdesc}
\begin{compactitem}
    \item Für $D \subset R, x_0 \in D, f,g: D \to \R$ stetig in $x_0 \implies |f|, \max(f,g), \min(f,g)$ stetig in $x_0$.
\end{compactitem}

\subsection{Min-Max Satz}
Für $f:I = [a, b] \to \R$ stetig auf kompaktem Intervall $I \implies \exists u,v \in [a,b], f(u) \le f(x) \le f(v) \ \forall x in [a,b] \iff f$ ist beschränkt.

\begin{inparaitem}
    \item $f(u) = \inf \{f(x) | x \in I\}$
    \item $f(v) = \sup \{f(x) | x \in I\}$.
\end{inparaitem}
\begin{compactdesc}
    \item[Kompakt Intervall:] Ist ein Intervall $I \subset \R$ falls $I = [a, b], a \le b$.
        \begin{compactitem}
        \item Für $(x_n)_{n \in \N}, \lim_{n \to \infty}x_n \in \R, a \le b$. Falls $\{x_n | n \ge 1\} \subset [a, b] \implies \lim_{n \to \infty}x_n \in [a, b]$.
        \end{compactitem}
\end{compactdesc}

\subsection{Umkehrabbildung}
Für $I \subset \R, f: I \to \R$ stetig und streng monoton $\implies J:=f(I) \subset \R, f^{-1}:  J \to I$ stetig und streng monoton.

\subsection{Reelle Exponentialfunktion}
$\exp: R \to ]o, +\infty[$ ist streng monoton wachsend, stetig und surjektiv.
\begin{compactitem}
    \item $\exp(x) = 1 + x + \frac{x^2}{2!} + \dots \ge 1$
    \item $\exp(x + y ) = \exp(x) \cdot \exp(y) \ \forall x,y \in \C$.
    \item $\exp(0) = 1$.
    \item $\exp(x) > 0 \ \forall x \in \R$
    \item $\exp(x) > 1 \ \forall x > 0$
    \item $\exp(x) > \exp(y) \ \forall x > y$
    \item $\exp(x) > 1 + x \ \exists x \in \R$
\end{compactitem}

\subsection{Natürliche Logarithmus}
Die Umkehrabbildung von $\exp$ ist $\ln: ]o, +\infty[ \to \R$ streng monoton wachsend, stetig und bijektiv.
\begin{compactitem}
    \item $\ln(1) = 0$
    \item $\ln(a \cdot b) = \ln(a) + \ln(b) \ \forall a,b \in ]o, +\infty[$
\end{compactitem}

\subsection{Allgemeine Potenzen}
Für $x > 0, a \in \R \ x^a := \exp(a \ln(x))$
\begin{compactitem}
    \item Für $a > 0$ ist $x \mapsto x^a$ stetig, streng monoton wachsend und bijektiv.
    \item Für $a < 0$ ist $x \mapsto x^a$ stetig, streng monoton fallend und bijektiv.
    \item $\ln(x^a) = a \ln(x) \ \forall a \in R, \ \forall x > 0$
    \item $x^a \cdot x ^b = x^{a + b} \ \forall a,b \in R, \ \forall x > 0$
    \item $(x^a)^b = x^{a \cdot b} \ \forall a,b \in R, \ \forall x > 0$
\end{compactitem}

\subsection{Funktionenfolgen}
\begin{compactdesc}
    \item[Funktionenfolge:] $\left( f_n \right)_{n \ge 0}$ ist eine Abbildung $\N \to \R^\mathbb{D} = \{f_n = \mathbb{D} \to R\}, n \mapsto f(n) = f_n$.
    \begin{compactitem}
        \item $\forall x \in \mathbb{D} \ \exists$ Folge $\left( f_n(x) \right)_{n \ge 0}$ in $\R$.
    \end{compactitem}
    \item[Konvergiert punktweise:] gegen Funktion $f: \mathbb{D} \to \R$ falls $\forall x \in \mathbb{D}: f(x) = \lim_{n \to \infty} f_n(x)$.
        \begin{compactitem}
            \item $f_n \overset{\text{p.w.}}{\to} f \notimplies f$ ist stetig
        \end{compactitem}
    \item[Konvergiert gleichmässig:] gegen Funktion $f: \mathbb{D} \to \R$ falls $\exists \epsilon > 0 \ \exists N \ge 1: \left| f_n(x) - f(x) \right| < \epsilon \ \forall n \ge N,\ \forall x \in \mathbb{D}$.
        \begin{compactitem}
            \item Für $\mathbb{D} \subset \R$ und Funktionenfolge $f_n:\mathbb{D} \to \R$ bestehend aus in $\mathbb{D}$ stetigen Funktionen die gleichmässig gegen Funktion $f: \mathbb{D} \to \R$ konvergieren $\implies f$ ist in $\mathbb{D}$ stetig.
            \item Falls $f_n$ gleichmässig zu $f$ konvergiert $\implies \limsup_{n \to \infty} \left| f_n(x) - f(x) \right| = 0, x \in \mathbb{D}$.
            \item $f_n \overset{\text{glm.}}{\to} f \implies f$ ist stetig
        \end{compactitem}
    \item[Gleichmässig konvergent:] falls $\forall x \in \mathbb{D} \ \exists f(x) = \lim_{n \to \infty} f_n(x)$ und $\left( f_n \right)_{n \ge 0}$ gleichmässig gegen $f$ konvergiert.
        \begin{compactitem}
            \item $f_n : \mathbb{D} \to \R$ ist gleichmässig konvergent $\iff \forall \epsilon > 0 \ \exists N \ge 1: \ \forall n,m \ge N \ \forall x \in D: \left| f_n(x) - f_m(x) \right| < \epsilon$.
            \item Falls $f_n:\mathbb{D} \to \R$ gleichmässig konvergente Folge stetiger Funktionen $\implies f(x) := \lim_{n \to \infty} f_n(x)$ stetig.
        \end{compactitem}
\end{compactdesc}

\subsubsection{Reihe von Funktionenfolgen}
$\sum_{k=0}^{\infty} f_k(x)$
\begin{compactdesc}
    \item[Konvergiert gleichmässig:] in $\mathbb{D}$ falls die Funktionenfolge $S_n(x) := \sum_{k=0}^{\infty} f_k(x)$ gleichmässig konvergiert.
    \item Für $\mathbb{D} \subset \R$, Folge stetiger Funktionen $f_n:\mathbb{D} \to \R$. Falls $\left| f_n(x) \right| \le c_n \ \forall x \in D$ und $\sum_{n=0}^{\infty} c_n$ konvergent $\implies \sum_{n=0}^{\infty} f_n(x)$ gleichmässig konvergent in $\mathbb{D}$ und Grenzwert $f(x) := \sum_{n=0}^{\infty} f_n(x)$ ist in $\mathbb{D}$ stetig.
\end{compactdesc}

\subsubsection{Potenzreihe}
\begin{compactitem}
    \item $\sum_{k=0}^{\infty} c_kx^k$
\end{compactitem}
\begin{compactdesc}
\item[Posiviten Konvergenzradius:] $\rho$ hat Potenzreihe falls $\limsup_{k \to \infty} \sqrt[^k]{\left| c_k \right| }$ existiert.
    \begin{compactitem}
        \item $\rho = \begin{cases}
            + \infty & \text{if } \limsup_{k \to \infty} \sqrt[^k]{\left| c_k \right| } = 0\\
            \frac{1}{\limsup_{k \to \infty}\sqrt[^k]{\left| c_k \right| }} & \text{if } \limsup_{k \to \infty} \sqrt[^k]{\left| c_k \right| } > 0
        \end{cases}$
    \end{compactitem}
\end{compactdesc}
\begin{compactitem}
    \item Für Potenzreihe mit positiven Konvergenzradius $\rho > 0$ und $f(x):= \sum_{n=1}^{\infty} c_kx^k, |x| < \rho \implies \forall 0 \le r < \rho$ konvergiert $\sum_{k=0}^{\infty} c_kx^k$ gleichmässig auf $[-r, r]$ und $f:]-\rho, \rho[ \to \R$ ist stetig.
    \item Sind stetig im Innern ihres Konvergenzbereiche
\end{compactitem}

\subsection{Trigonometrische Funktionen}
\begin{compactitem}
    \item $\sin(z) = z - \frac{z^2}{3!} + \frac{z^{5}}{5!} - \dots = \sum_{n=0}^{\infty} \frac{(-1)^n z^{2n + 1}}{(2n + 1)!}$.
        \begin{compactitem}
            \item $\sin: \R \to \R$ stetig.
        \end{compactitem}
    \item $\cos(z) = 1 - \frac{z^2}{2!} + \frac{z^{4}}{4!} - \dots = \sum_{n=0}^{\infty} \frac{(-1)^nz^{2n}}{(2n)!}$.
        \begin{compactitem}
            \item $\cos: \R \to \R$ stetig.
        \end{compactitem}
\end{compactitem}
\begin{compactenum}
    \item $\exp(iz) = \cos(x) + i \sin(z) \ \forall z \in \C$
    \item
        \begin{inparaitem}
            \item $\cos(z) = \cos(-z)$
            \item $\sin(-z) = - \sin(z) \ \forall z \in \C$
        \end{inparaitem}
    \item
        \begin{inparaitem}
            \item $\sin(z) = \frac{e^{iz} - e^{-iz}}{2i}$
            \item $\cos(z) = \frac{e^{iz} + e^{-iz}}{2}$
        \end{inparaitem}
    \item
        \begin{inparaitem}
            \item $\sin(z + w) = \sin(z) \cos(w) + \cos(z) \sin(w)$
            \item $\cos(z + w) = \cos(z) \cos(w) - \sin(z) \sin(w)$
        \end{inparaitem}
    \item $\cos(z)^2 + \sin(z)^2 = 1 \ \forall z \in \C$
    \item
        \begin{inparaitem}
            \item $\sin(2z) = 2 \sin(z) \cos(z)$
            \item $\cos(2z) = \cos(z)^2 - \sin(z)^2$
        \end{inparaitem}
\end{compactenum}

\subsection{Kreiszahl}
\begin{compactitem}
    \item $\sin(0) = 0$
    \item $\sin$ hat auf $]0, +\infty[$ min. eine Nullstelle.
    \item für $\pi := \inf \{t > 0 | \sin(t) = 0\} \implies$
        \begin{compactenum}
            \item $\sin(\pi) = 0 \ \pi \in ]2, 4[$
            \item $\forall x \in ]0, \pi[ : \sin(x) > 0$
            \item $e^{\frac{i\pi}{2}} = i$
        \end{compactenum}
    \item $x \ge \sin(x) \ge x - \frac{x^3}{3!} \ \forall 0 \le x \le \sqrt{6}$
\end{compactitem}

$\forall x \in \R$
\begin{compactenum}
    \item
        \begin{inparaitem}
            \item $e^{i \pi} = -1$
            \item $e^{2 \pi i} = 1$
        \end{inparaitem}
    \item
        \begin{inparaitem}
            \item $\sin(x + \frac{\pi}{2}) = \cos(x)$
            \item $\cos(x + \frac{\pi}{2}) = - \sin(x)$
        \end{inparaitem}
    \item
        \begin{inparaitem}
            \item $\sin(x + \pi) = - \sin(x)$
            \item $\cos(x + \pi) = - \cos(x)$
        \end{inparaitem}
    \item
        \begin{inparaitem}
            \item $\sin(x + 2 \pi) = \sin(x)$
            \item $\cos(x + 2 \pi) = \cos(x)$
        \end{inparaitem}
    \item Nullstellen von:
        \begin{compactitem}
            \item $\sin(x) = \{k \cdot \pi | k \in \Z\}$
                \begin{inparaitem}
                    \item $\sin(x) > 0, \ \forall x \in ] 2k \pi, (2k+1) \pi[$ 
                    \item $\sin(x) < 0, \ \forall x \in ](2k + 1) \pi, (2k+2) \pi[$
                \end{inparaitem}
            \item $\cos(x) = \{\frac{\pi}{2} + k \cdot \pi | k \in \Z\}$
                \begin{inparaitem}
                \item $\cos(x) > 0, \ \forall x \in ]\frac{-\pi}{2} + 2k\pi, \frac{-\pi}{2} + (2k+1) \pi[$
                \item $\cos(x) < 0, \ \forall x \in ]\frac{-\pi}{2} + (\frac{2k + 1}{2}\pi, \frac{-\pi}{2} + (2k+2) \pi[$
                \end{inparaitem}
        \end{compactitem}
\end{compactenum}

\begin{compactdesc}
    \item[Tangens:] $\tan(z) := \frac{\sin(z)}{\cos(z)}, z \not\in \{\frac{\pi}{2} + \pi k\}$
    \item[Cotangens:] $\cot(z) := \frac{\cos(z)}{\sin(z)}, z \not\in \{\pi k\}$
\end{compactdesc}


\subsection{Grenzwert von Funktionen}
Function $f: \mathbb{D} \to \R, \mathbb{D} \subset \R$. Grenzwert 
\begin{compactdesc}
    \item[Häufigkeitspunkt:] von $\mathbb{D}$ falls $\forall \delta > 0 (]x_0 - \delta, x_0 + \delta[ \setminus \{x_0\} ) \cap \mathbb{D} \neq \O$.
    \item[Grenzwert:] $A \in \R$ von $f(x)$ für $x \to x_0$ und Funktion $f:\mathbb{D} \to \R$ und Häufigkeitspunkt $x_0 \in \R$ von $\mathbb{D}$. Wird mit $\lim_{n \to \infty} f(x) = A$ bezeichnet falls $\forall \epsilon > 0 \ \exists \delta > 0$ so dass $\forall x \in \mathbb{D} \cap (]x_0 - \delta, x_0 + \delta[ \setminus \{x_0\}) : |f(x) - A| < \epsilon$.
\end{compactdesc}
\begin{compactitem}
    \item $f$ muss am Grenzwert $x_0$ nicht zwingend definiert sein.
    \item Für $f$ und Häufigkeitspunkt $x_0$. $\lim_{n \to \infty}f(x) = A \iff \forall (a_n)_{n \ge 1} \in \mathbb{D} \setminus \{x_0\}$ mit $\lim_{n \to \infty} a_n = x_0$ folgt $\lim_{n \to \infty}f(a_n) = A$.
    \item $f$ ist stetig in $x_0 \in \mathbb{D} \iff \lim_{n \to \infty} f(x) = f(x_0)$.
    \item Für $f,g: \mathbb{D} \to \R$ und falls $\exists \lim_{n \to \infty} f(x)$ und $\lim_{n \to \infty} g(x) \implies$:
        \begin{inparaitem}
            \item $\lim_{n \to \infty} (f + g)(x) = \lim_{n \to \infty} f(x) + \lim_{n \to \infty} g(x)$
            \item $\lim_{n \to \infty} (f \cdot g)(x) = \lim_{n \to \infty} f(x) \cdot \lim_{n \to \infty} g(x)$
        \end{inparaitem}
    \item Für $f,g: \mathbb{D} \to \R, f \le g$ und beide Grenzwerte existieren $\implies \lim_{n \to \infty} f(x) \le \lim_{n \to \infty} g(x)$.
    \item Für $f,g_1, g_2: \mathbb{D} \to \R$, falls $g_1 \le f \le g_2$ und $\lim_{n \to \infty} g_1(x) = \lim_{n \to \infty} g_2(x) \implies \exists \lim_{n \to \infty} f(x)$ und $\lim_{n \to \infty} f(x) = \lim_{n \to \infty} g_1(x)$.
\end{compactitem}

\subsubsection{Links- und rechtsseitige Grenzwerte}
\begin{compactdesc}
    \item[Rechtsseitiger Grenzwert:] $\lim_{x \to x_0^+}$ falls der Grenzwert der eingeschränkten Funktion $f|_{\mathbb{D} \cap [x_0, +\infty[}$ für $x \to x_0$ existiert. Wobei $f:\mathbb{D} \to \R$, $x_0 \in \R$ ein Häufigkeitspunkt von $\mathbb{D} \cap ]x_0, + \infty[$.
    \item[Linksseitiger Grenzwert:] $\lim_{x \to x_0^-}$ falls der Grenzwert der eingeschränkten Funktion $f|_{\mathbb{D} \cap [-\infty, x_0[}$ für $x \to x_0$ existiert. Wobei $f:\mathbb{D} \to \R$, $x_0 \in \R$ ein Häufigkeitspunkt von $\mathbb{D} \cap ]x_0, + \infty[$.
    \item[Erweitert Rechts:] $\lim_{x \to x_0^+} f(x) = +\infty$ falls $\exists \epsilon > 0, \ \exists \delta > 0, \ \forall x \in ]x_0, x_0 + \delta[ \cap : f(x) > \frac{1}{\epsilon}$.
        \begin{compactitem}
        \item Alternativ: $\forall N > 0, \ \exists \delta > 0 \ \forall x \in D \cap ]x_0, x_0 + \delta[ : f(x) > N$.
        \end{compactitem}
    \item[Erweitert Links:] $\lim_{x \to x_0^-} f(x) = -\infty$ falls $\exists \epsilon > 0, \ \exists \delta > 0, \ \forall x \in ]x_0, x_0 + \delta[ \cap : f(x) < -\frac{1}{\epsilon}$.
        \begin{compactitem}
        \item Alternativ: $\forall N > 0, \ \exists \delta > 0 \ \forall x \in D \cap ]x_0, x_0 + \delta[ : f(x) < -N$.
        \end{compactitem}
\end{compactdesc}

\subsubsection{Unendlicher Grenzwert}
\begin{compactdesc}
    \item[Oben:] Für $f: \mathbb{D} \to \R^n, \mathbb{D}$ nach oben beschränkt, so ist $\lim_{x \to +\infty} f(x) = L \in \R$ falls $\forall \epsilon > 0, \ \exists c > 0: \ \forall x \in D, \ x > c \implies |f(x) - L| < \epsilon$
    \item[Unten] Für $f: \mathbb{D} \to \R^n, \mathbb{D}$ nach unten beschränkt, so ist $\lim_{x \to -\infty} f(x) = L \in \R$ falls $\forall \epsilon > 0, \ \exists c > 0: \ \forall x \in D, \ x < -c \implies |f(x) - L| < \epsilon$
\end{compactdesc}

    \todo{Bisektionsverfahren}
\end{multicols}
\end{document}
