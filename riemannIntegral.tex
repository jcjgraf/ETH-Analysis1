%! TEX root  = ./main.tex

\section{Riemann Integral}
\subsection{Partition}
$P$ von Interval $I$ ist eine endliche Teilmenge $P \subset [a, b], a < b \in \R$ mit $a \in P, b \in P$.
\begin{inparaitem}
    \item $\delta_i := x_i - x_{i - 1}$
    \item $\mathbb{P}(I)$ Menge aller $P \subset I$.
    \item $\mathbb{P}_\delta(I)$ Menge aller $P \subset I, \max_{1 \le i \le n} \delta_i \le \delta$.
\end{inparaitem}
\begin{compactdesc}
    \item[Feinheit:] $\delta(P) := max_{1 \le i \le n} \delta_i$.
    \item[Verfeinerung:] $P'$ von $P$ falls $P' \subset P$.
    \item[Zwischen Punkte:] $\xi_i \in I_i$.
\end{compactdesc}

\subsection{Riemannsche Summe}
\begin{compactdesc}
\item[Riemannsche Summe:] $S(f, P, \xi) := \sum_{i=1}^{n} f(\xi_i) \delta_i$.
    \item[Untersumme:] $s(f, P) := \sum_{i=1}^{n} \delta_i \cdot \inf_{x_{i-1} \le x \le x_i} f(x)$.
    \item[Obersumme:] $S(f, P) := \sum_{i=1}^{n} \delta_i \cdot \sup_{x_{i-1} \le x \le x_i} f(x)$.
        \begin{compactitem}
            \item $s(f, P_1) \le s(f, P_2) \ \forall P_1, P_2 \subset I$.
            \item $s(f, p) \le s(f, P') \le S(f, P') \le S(f, P) \ \forall P' \subset P \subset I$.
        \end{compactitem}
    \item[Unteres Integral:] $s(f) := \sup_{P \in \mathbb{P}} s(f, P)$.
    \item[Oberes Integral:] $S(f) := \inf_{P \in \mathbb{P}} S(f, P)$.
        \begin{compactitem}
        \item $s(f) \le S(f)$.
        \end{compactitem}
\end{compactdesc}

\subsection{Riemann-integrierbar}
Beschränkte Funktion $f:[a,b] \to R$ ist integrierbar
\begin{compactitem}
    \item falls $s(f) = S(f) =: \int_{a}^{b} f(x) \mathrm{d}x$.
    \item $\iff \forall \epsilon > 0 \ \exists P \in \mathbb{P}, S(f, P) - s(f, P) < \epsilon$.
    \item $\iff \forall \epsilon > 0 \ \exists \delta > 0: \ \forall P \in \mathbb{P}_\delta(I), S(f, P) - s(f, P) < \epsilon$.
    \item mit $A:= \int_{a}^{b} f(x) \mathrm{d}x \iff \forall \epsilon > 0 \ \exists \delta > 0: \ \forall P \in \mathbb{P}_\delta(I), \ \xi_i \in [x_{i-1}, x_i], \ \left| A - \sum_{i=1}^{n} f(\xi_i) \delta_i \right| < \epsilon$.
    \item $\iff \exists \lim_{\delta{P} \to 0} S(f, P, \xi) =: \int_{a}^{b} f(x) \mathrm{d}x$.
\end{compactitem}

\subsection{Funkionenverknüpfung}
\begin{compactitem}
    \item Für $f,g: [a, b] \to \R$ beschränkt und integrierbar, $\lambda \in \R \implies$
        \begin{inparaitem}
            \item $f+g$
            \item $\lambda * f$
            \item $f \cdot g$
            \item $|f|$
            \item $\max(f, g)$
            \item $\min(f, g)$
            \item $f / g$
        \end{inparaitem} integrierbar.
    \item Für Polynome $P, Q$ und Intervall $[a, b]$, $G$ hat keine Nullstelle $\implies [a, b] \to \R, x \mapsto P(x) / Q(x)$ integrierbar.
    \item $f: [a, b] \to \R$ stetig $\implies f$ integrierbar.
    \item $f: [a, b] \to \R$ monoton $\implies f$ integrierbar.
    \item $a < b < c, f: [a, c] \to \R$ beschränkt mit $f|_{[a,b]}$ und $f|_{[b,c]}$ integrierbar $\implies f$ integrierbar und $\int_{a}^{c} f(x) \mathrm{d}x = \int_{a}^{b} f(x) \mathrm{d}x + \int_{b}^{c} f(x) \mathrm{d}x$.
    \item
        \begin{inparaitem}
            \item $\int_{a}^{a} f(x)\mathrm{d}x = 0$
            \item $\int_{a}^{b} f(x)\mathrm{d}x = -\int_{b}^{a} \mathrm{d}x$
        \end{inparaitem}
    \item Für kompaktes Intervall $I \subset \R$ mit Endpunkten $a,b$, Funktionen $f, g : I \to \R$ beschränkt, integrierbar und $\alpha, \beta \in \R \implies \int_{a}^{b} (\alpha f(x) + \beta g(x)) \mathrm{d}x = \alpha \int_{a}^{b} f(x) \mathrm{d}x + \beta \int_{a}^{b} g(x)\mathrm{d}x$.
\end{compactitem}

\subsection{Eigenschaften}
Für $f, g: [a,b] \to \R$ beschränkt integrierbar:
\begin{compactitem}
    \item falls $f(x) \le g(x) \ \forall x \in [a,b] \implies \int_{a}^{b} f(x) \mathrm{d}x \le \int_{a}^{b} g(x) \mathrm{d}x$.
    \item $\left| \int_{a}^{b} f(x) \mathrm{d}x \right| \le \int_{a}^{b} \left| f(x) \right| \mathrm{d}x$.
    \item $\left| \int_{a}^{b} f(x)g(x) \mathrm{d}x \right| \le \sqrt{\int_{a}^{b} f^2(x) \mathrm{d}x}\sqrt{\int_{a}^{b} g^2(x) \mathrm{d}x}$.
    \item Für Intervall $I \subset \R$ und $f: I \to \R$ stetig:
        \begin{compactitem}
            \item Für $a,b,c \in \R,$ Intervall $[a+c, b+c] \in I \implies \int_{a+c}^{b+c} f(x) \mathrm{d}x = \int_{a}^{b} f(t + c)\mathrm{d}t$.
            \item Für $a,b,c \in R, c \neq 0,$ Intervall $[a \cdot c, b \cdot c] \in I \implies \int_{a}^{b} f(c \cdot t) \mathrm{d}t = \frac{1}{c} \int_{ac}^{bc} f(x) \mathrm{d}x$.
        \end{compactitem}
\end{compactitem}

\subsection{Mittelwertsatz}
Für $f:[a,b] \to \R$ stetig. $\implies \exists \xi \in [a,b], \int_{a}^{b} f(x) \mathrm{d}x = f(\xi)(b-a)$.
\begin{compactitem}
\item für $f,g :[a,b] \to \R, f$ stetig, $g$ beschränkt integrierbar und $g(x) \ge 0 \ \forall x \in [a,b] \implies \exists \xi \in [a,b], \int_{a}^{b} f(x)g(x) \mathrm{d}x = f(\xi) \int_{a}^{b} g(x)\mathrm{d}x$.
\end{compactitem}

\subsection{Fundamentalsatz der Differentialrechnung}
Für $f:[a,b] \to \R$ stetig $\implies \exists$ Stammfunktion $F$ von $f$, welche bis auf eine additive Konstante eindeutig ist und $\int_{a}^{b} f(x) \mathrm{d}x = F(b) - F(a)$.
\begin{compactitem}
    \item Für $a < b, \ f:[a,b] \to \R$ stetig $\implies F(x) = \int_{a}^{x} f(t) \mathrm{d}t, \ a \le x \le b$ in $[a, b]$ stetig differenzierbar und $F'(x) = f(x) \ \forall x \in [a, b]$.
    \item Für $a < b, \ f:[a,b] \to \R$ stetig. Die \textbf{Stammfunktion} $F: [a,b] \to \R$ von $f$ ist differenzierbar in $[a,b]$ und $F'=f$.
\end{compactitem}

\subsection{Partielle Integration}
Für $a < b,\ f,g: [a,b] \to \R$ stetig differenzierbar $\implies \int_{a}^{b} f(x)g'(x) \mathrm{d}x = f(b)g(b) - f(a)g(a) - \int_{a}^{b} f'(x)g(x) \mathrm{d}x  = f(x)g(x)|_a^b - \int_{a}^{b} f'(x)g(x) \mathrm{d}x$.

\subsection{Substitution}
Für $a < b, \ \phi: [a, b] \to \R$ stetig differenzierbar, Intervall $I \subset \R$ mit $\phi([a,b]) \subset I$ und $f: I \to \R$ stetig $\implies \int_{\phi(a)}^{\phi(b)} f(x) \mathrm{d}x = \int_{a}^{b} f(\phi(t)) \phi'(t) \mathrm{d}t$.
\begin{compactitem}
    \item Für unbestimmtes integral $\int f(x) \mathrm{d}x |_{x=\phi(t)} = \int f(\phi(t))\phi'(t) \mathrm{d}t + c$.
    \item Integral in der Form $\int_{t_0}^{t_1} f(\phi(t))\phi'(t)\mathrm{d}t \implies$ Anwendung von links nach rechts.
    \item Integral in der Form $\int_{\alpha}^{\beta} f(x) \mathrm{d}x \implies$ versuch einer Substitution mittels $x = \phi(t)$, wobei $\phi(t_0) = \alpha$ und $\phi(t_1) = \beta$.
    \item Für best. Integrale gibt es zwei Methoden für den Umgang mit Grenzwerten:
        \begin{compactitem}
            \item Substitution von $x=\phi(t)$, Brechung einer Stammfunktion in $x$ und ersetzten der Variable $x$ mit $t$ und Benutzung des Grenze für $t$.
            \item Änderung der Grenze während der Substitution.
        \end{compactitem}
\end{compactitem}

\subsection{Konvergente Reihen}
\begin{compactitem}
    \item Für beschränkte, integrierbare Folge von Funktionen $f_n: [a,b] \to \R$ welche gleichmässig gegen Funktion $f:[a,b] \to R$ konvergieren $\implies f$ ist beschränkt integrierbare und $\lim_{n \to \infty} \int_{a}^{b} f_n(x) \mathrm{d}x = \int_{a}^{b} f(x) \mathrm{d}x$.
    \item Für Folge $f_n [a,b] \to \R$ beschränkter integrierbarer Funktionen, $\sum_{n=0}^{\infty} f_n$ auf $[a,b]$ gleichmässig konvergiert $\implies \sum_{n=0}^{\infty} \int_{a}^{b} f_n(x) \mathrm{d}x = \int_{a}^{b} \left( \sum_{n=0}^{\infty} f_n(x) \right) \mathrm{d}x$.
    \item Für Potenzreihe $f(x) = \sum_{n=0}^{\infty} c_kx^k$ mit positiven Konvergenzradius $\rho > 0 \implies \forall 0 \le r < \rho$ ist $f$ auf $[-r, r]$ integrierbar und $\forall x \in ]-\rho, \rho[ \ \int_{0}^{x} f(t) \mathrm{d}t = \sum_{n=0}^{\infty} \frac{c_n}{n + 1} x^{n+1}$.
        \begin{compactitem}
            \item Potenzreihen können auf ihrem Konvergenzbereich gliedweise differenziert und integriert werden.
        \end{compactitem}
\end{compactitem}

\todo{Approximation von Summen}

\subsection{Uneigentliche Integrale}
\begin{compactitem}
    \item Für $f: [a, \infty] \to \R$ beschränkt und $\forall a < b$ auf $[a,b]$ integrierbar. Falls $\exists \lim_{b \to \infty} \int_{a}^{b} f(x) \mathrm{d}x \implies$ Grenzwert ist $\int_{a}^{\infty} f(x) \mathrm{d}x$ und $f$ ist auf $[a, +\infty[$ differenzierbar.
    \begin{compactitem}
        \item Fall für $f:[-\infty, b] \to \R$ verläuft analog.
    \end{compactitem}
    \item Für $f:[a, \infty[ \to \R$ beschränkt und $\forall a < b$ auf $[a,b]$ integrierbar:
        \begin{compactitem}
            \item falls $|f(x)| \le g(x) \ \forall x \ge a$ und $g(x)$ integrierbar auf $[a, +\infty[ \implies f$ auf $[a, \infty[$ integrierbar.
            \item falls $0 \le g(x) \le f(x)$ und $\int_{a}^{\infty} g(x) \mathrm{d}x$ divergent $\implies \int_{a}^{\infty} f(x) \mathrm{d}x$ divergent.
        \end{compactitem}
    \item Für $f: [1, \infty[ \to [0, \infty[$ monoton fallend. Die Reihe $\sum_{n=1}^{\infty} f(n)$ konvergiert $\iff \int_{1}^{\infty} f(x) \mathrm{d}x$ konvergiert und in diesem Fall ist $0 \le \sum_{k=1}^{\infty} f(k) - \int_{1}^{\infty} f(x) \mathrm{d}x \le f(1)$.
    \item Funktion $f$ auf $[a + \epsilon, b]$ beschränkt und integrierbar. Dann ist $f:]a,b] \to \R$ integrierbar falls $\exists \lim_{\epsilon \to 0^+} \int_{a + \epsilon}^{b} f(x) \mathrm{d}x$. In diesem Fall ist der Grenzwert $\int_{a}^{b} f(x) \mathrm{d}x$.
    \begin{compactitem}
        \item Fall für $f:[a, b - \epsilon] \to \R$ verläuft analog.
    \end{compactitem}
    \item $\int_{-\infty}^{\infty} f(x) \mathrm{d}x := \int_{-\infty}^{c} f(x) \mathrm{d}x + \int_{c}^{\infty} f(x) \mathrm{d}x$.
    \item Für beidseitig offene Intervalle müssen mir den Grenzwert unabhängig nehmen: $\int_{-\infty}^{\infty} f(x) \mathrm{d}x = \lim_{a \to -\infty} \lim_{b \to +\infty} \int_{a}^{b} f(x) \mathrm{d}x$.
\end{compactitem}

\subsubsection{Gamma Funktion}
$\Gamma(s):= \int_{0}^{\infty} e^{-x}x^{s-1} \mathrm{d}x = (n-1)!$.

Folgende Eigenschaften beschreiben die Gamma Funktion einzigartig.
\begin{inparaenum}
    \item $\Gamma(1) = 1$
    \item $\gamma(s + 1) = s \Gamma(s) \ \forall s > 0$
    \item $\Gamma(s)$ ist logarithmisch konvex: $\Gamma(\lambda x + (1- \lambda)y) \le \Gamma(x)^\lambda \Gamma(y)^{1-\lambda} \ \forall x,y > 0, \ 0 \le \lambda \le 1$
\end{inparaenum}
\begin{compactitem}
    \item $\int_{-\infty}^{\infty} e^{-t^2}\mathrm{d}t = \Gamma(\frac{1}{2})$
\end{compactitem}

\subsection{Rationale Funktionen}
Integration von $R(x) = P(x) / Q(x), \ P(x), Q(x)$ sind Polynome:
\begin{compactitem}
    \item 
\end{compactitem}
