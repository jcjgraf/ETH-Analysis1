%! TEX root = ./main.tex

\section{Stetige Funktionen}
\subsection{Reellwertige Funktionen}
Für beliebige Menge $D$ ist $\R^D = \{f : D \to \R | f \text{ eine Abbildung}\}$ die Menge alle reellwertigen Funktionen die auf $D$ definiert sind.\\
Addition und skalare Multiplikation bilden mit $\R^D$ einen Vektorraum. Für $f, g \in \R^D, x \in D, \alpha \in \R$:
\begin{compactdesc}
    \item[Addition:] $(f + g)(x) = f(x) + g(x)$
    \item[Skalare Multiplikation:] $\left( \alpha \cdot f \right) (x) = \alpha \cdot f(x)$
    \item[Nullfunktion:] Entspricht dem Nullvektor in $\R^D$ und $\zeta(x) = 0$
    \item[Constate Funkton:]  Entspricht dem Einheitsvektor in $\R^D$ und $\zeta(x) = 1$
    \item[Produkt zweier Funktionen:] $(f \cdot g)(x) := f(x) \cdot g(x)$
    \item[Quotient:] $\frac{f}{g} := D' \to \R, x \mapsto \frac{f(x)}{g(x)}, \ D'= \{x \in D | g(x) \neq 0\} $
    \item[Komposition von Funktionen:] $f: D \to \R$ und $f: E \to \R, f(D) \subset E$ dann $g \circ f: D \to \R, x \mapsto g\left( f(x) \right)$
\end{compactdesc}

\subsection{Beschränktheit}
$f : D \to \R$ ist:
\begin{compactdesc}
    \item[nach oben beschränkt:] falls $f(D) \subset \R$ nach oben beschränkt ist.
    \item[nach unten beschänkt:] falls $f(D) \subset \R$ nach unten beschränkt ist.
    \item[beschränkt:] falls $f(D) \subset \R$ beschränkt ist.
\end{compactdesc}

\subsection{Monotonie}
$f : D \to \R, D \subset R, \ \forall x,y \in D$ ist:
\begin{compactdesc}
    \item[monoton wachsend:] falls $x \le y \implies f(x) \le f(y)$.
    \item[streng monoton wachsend:] falls $x < y \implies f(x) < f(y)$.
    \item[monoton fallend:] falls $x \ge y \implies f(x) \ge f(y)$.
    \item[streng monoton fallend:] falls $x > y \implies f(x) > f(y)$.
    \item[monoton:] falls monoton wachsend oder monoton fallend.
    \item[streng monoton:] falls streng monoton wachsend oder streng monoton fallend.
\end{compactdesc}
\todo{Show monotony}

\subsection{Stetigkeit}
\begin{compactdesc}
    \item[$\mathbf{x_0}$ stetig:] $f: D \to \R$ für $D \subset \R, x_o \in D$  falls  $\forall \epsilon > 0 \ \exists \delta > 0: \left| x - x_0 \right| < \delta \implies \left| f(x) - f_0(x) \right| < \epsilon \ \forall x \in D $.
    \item[stetig:] $f: D \to \R$ falls sie in jedem Punkt von $D$ stetig ist.
\end{compactdesc}

\begin{compactitem}
    \item $f$ ist in $x_0$ stetig $\iff \forall (a_n)_{n \in \N}$ in $D: \ \lim_{n \to \infty} a_n = x_0 \implies \lim_{n \to \infty} f(a_n) = f(x_0)$.
        \begin{compactitem}
            \item $f$ ist in $x_0$ stetig $\iff \lim_{n \to \infty} f(a_n) = f( \lim_{n \to \infty} a_n) \ \forall (a_n)_{n \in \N}$ in $D$.
        \end{compactitem}
\end{compactitem}

\subsection{Rechenregeln}
Für $x_0 \in D \subset \R, \lambda \in \R, f: D \to \R, g: D \to \R$ und $f$ und $g$ stetig in $x_0 \implies$
\begin{compactenum}
    \item $f + g, \lambda \cdot f, f \cdot g$ stetig in $x_0$.
    \item $\frac{f}{g}: D' \to \R, x \mapsto \frac{f(x)}{g(x)}, \ D' = \{x \in D | g(x) \neq 0\}, g(x_0) \neq 0$ ist stetig in $x_0$.
\end{compactenum}
Für $D_1, d_2 \subset \R, f: D_1 \to D_2, g: D_2 \to \R, x_0 \in D_1$. Falls $f$ in $x_0$ und $g$ in $f(x_0)$ stetig $\implies g \circ f : D_1 \to \R$ ist in $x_0$ stetig.
\begin{compactitem}
    \item Falls $f$ auf $D_1$ und $g$ auf $D_2$ stetig $\implies g \circ f$ auf $D_1$ stetig.
\end{compactitem}

\subsection{Polynom}
Funktion $P: \R \to \R, P(x)= a_nx^n + \dots + a_0, a_n, \dots a_0 \in \R$.
\begin{compactdesc}
    \item[Grad:] ist $n$ falls $a_n \neq 0$.
\end{compactdesc}
\begin{compactitem}
    \item Sind auf ganz $\R$ stetig.
    \item $\frac{P}{Q}: R \setminus \{x_1, \dots x_m\}  \to \R, \ x \mapsto \frac{P(x)}{Q(x)}$ ist stetig für $P,Q$ auf $\R$ wobei $ Q \neq 0$  und Nullstellen $x_1, \dots, x_m$ von $Q$.
\end{compactitem}

\subsection{Zwischerwertsatz}
Für Intervall $I \subset R$, stetige Funktion $f: I \to \R$ und $a, b \in I \implies \forall c$ zwischen $f(a)$ und $f(b) \ \exists z$ zwischen $a$ und $b$ mit $f(z) = c$.
\begin{compactitem}
    \item $x, y \in R, x \le y$, $c$ liegt \textbf{zwischen} $x$ und $y$ falls $c \in [x, y]$.
    \item Ein Polynom $P$ mit ungeradem Grad $n$ besitzt mindestens eine Nullstelle in $\R$.
    \item Für $f: [a,b] \to \R$ stetig und $f(a) \cdot f(b) < 0 \implies \exists c \in ]a,b[: f(c) = 0$.
\end{compactitem}

\subsection{Min, Max, Abs}
Für menge $D$ und $f,g: D \to \R$:
\begin{compactdesc}
    \item[Abs:] $|f|(x) := |f(x)|, \ \forall x \in D$
    \item[Max:] $\max(f,g)(x) := \max(f(x), g(x)), \ \forall x \in D$
    \item[Min:] $\min(f,g)(x) := \min(f(x), g(x)), \ \forall x \in D$
\end{compactdesc}
\begin{compactitem}
    \item Für $D \subset R, x_0 \in D, f,g: D \to \R$ stetig in $x_0 \implies |f|, \max(f,g), \min(f,g)$ stetig in $x_0$.
\end{compactitem}

\subsection{Min-Max Satz}
Für $f:I = [a, b] \to \R$ stetig auf kompaktem Intervall $I \implies \exists u,v \in [a,b], f(u) \le f(x) \le f(v) \ \forall x in [a,b] \iff f$ ist beschränkt.

\begin{inparaitem}
    \item $f(u) = \inf \{f(x) | x \in I\}$
    \item $f(v) = \sup \{f(x) | x \in I\}$.
\end{inparaitem}
\begin{compactdesc}
    \item[Kompakt Intervall:] Ist ein Intervall $I \subset \R$ falls $I = [a, b], a \le b$.
        \begin{compactitem}
        \item Für $(x_n)_{n \in \N}, \lim_{n \to \infty}x_n \in \R, a \le b$. Falls $\{x_n | n \ge 1\} \subset [a, b] \implies \lim_{n \to \infty}x_n \in [a, b]$.
        \end{compactitem}
\end{compactdesc}

\subsection{Umkehrabbildung}
Für $I \subset \R, f: I \to \R$ stetig und streng monoton $\implies J:=f(I) \subset \R, f^{-1}:  J \to I$ stetig und streng monoton.

\subsection{Reelle Exponentialfunktion}
$\exp: R \to ]o, +\infty[$ ist streng monoton wachsend, stetig und surjektiv.
\begin{compactitem}
    \item $\exp(x) = 1 + x + \frac{x^2}{2!} + \dots \ge 1$
    \item $\exp(x + y ) = \exp(x) \cdot \exp(y) \ \forall x,y \in \C$.
    \item $\exp(0) = 1$.
    \item $\exp(x) > 0 \ \forall x \in \R$
    \item $\exp(x) > 1 \ \forall x > 0$
    \item $\exp(x) > \exp(y) \ \forall x > y$
    \item $\exp(x) > 1 + x \ \exists x \in \R$
\end{compactitem}

\subsection{Natürliche Logarithmus}
Die Umkehrabbildung von $\exp$ ist $\ln: ]o, +\infty[ \to \R$ streng monoton wachsend, stetig und bijektiv.
\begin{compactitem}
    \item $\ln(1) = 0$
    \item $\ln(a \cdot b) = \ln(a) + \ln(b) \ \forall a,b \in ]o, +\infty[$
\end{compactitem}

\subsection{Allgemeine Potenzen}
Für $x > 0, a \in \R \ x^a := \exp(a \ln(x))$
\begin{compactitem}
    \item Für $a > 0$ ist $x \mapsto x^a$ stetig, streng monoton wachsend und bijektiv.
    \item Für $a < 0$ ist $x \mapsto x^a$ stetig, streng monoton fallend und bijektiv.
    \item $\ln(x^a) = a \ln(x) \ \forall a \in R, \ \forall x > 0$
    \item $x^a \cdot x ^b = x^{a + b} \ \forall a,b \in R, \ \forall x > 0$
    \item $(x^a)^b = x^{a \cdot b} \ \forall a,b \in R, \ \forall x > 0$
\end{compactitem}

\subsection{Funktionenfolgen}
\begin{compactdesc}
    \item[Funktionenfolge:] $\left( f_n \right)_{n \ge 0}$ ist eine Abbildung $\N \to \R^\mathbb{D} = \{f_n = \mathbb{D} \to R\}, n \mapsto f(n) = f_n$.
    \begin{compactitem}
        \item $\forall x \in \mathbb{D} \ \exists$ Folge $\left( f_n(x) \right)_{n \ge 0}$ in $\R$.
    \end{compactitem}
    \item[Konvergiert punktweise:] gegen Funktion $f: \mathbb{D} \to \R$ falls $\forall x \in \mathbb{D}: f(x) = \lim_{n \to \infty} f_n(x)$.
        \begin{compactitem}
            \item $f_n \overset{\text{p.w.}}{\to} f \notimplies f$ ist stetig
        \end{compactitem}
    \item[Konvergiert gleichmässig:] gegen Funktion $f: \mathbb{D} \to \R$ falls $\exists \epsilon > 0 \ \exists N \ge 1: \left| f_n(x) - f(x) \right| < \epsilon \ \forall n \ge N,\ \forall x \in \mathbb{D}$.
        \begin{compactitem}
            \item Für $\mathbb{D} \subset \R$ und Funktionenfolge $f_n:\mathbb{D} \to \R$ bestehend aus in $\mathbb{D}$ stetigen Funktionen die gleichmässig gegen Funktion $f: \mathbb{D} \to \R$ konvergieren $\implies f$ ist in $\mathbb{D}$ stetig.
            \item Falls $f_n$ gleichmässig zu $f$ konvergiert $\implies \limsup_{n \to \infty} \left| f_n(x) - f(x) \right| = 0, x \in \mathbb{D}$.
            \item $f_n \overset{\text{glm.}}{\to} f \implies f$ ist stetig
        \end{compactitem}
    \item[Gleichmässig konvergent:] falls $\forall x \in \mathbb{D} \ \exists f(x) = \lim_{n \to \infty} f_n(x)$ und $\left( f_n \right)_{n \ge 0}$ gleichmässig gegen $f$ konvergiert.
        \begin{compactitem}
            \item $f_n : \mathbb{D} \to \R$ ist gleichmässig konvergent $\iff \forall \epsilon > 0 \ \exists N \ge 1: \ \forall n,m \ge N \ \forall x \in D: \left| f_n(x) - f_m(x) \right| < \epsilon$.
            \item Falls $f_n:\mathbb{D} \to \R$ gleichmässig konvergente Folge stetiger Funktionen $\implies f(x) := \lim_{n \to \infty} f_n(x)$ stetig.
            \item Alle $f_n$ stetig und $f_n \to f \implies f$ ist stetig.
        \end{compactitem}
\end{compactdesc}

\subsubsection{Reihe von Funktionenfolgen}
$\sum_{k=0}^{\infty} f_k(x)$
\begin{compactdesc}
    \item[Konvergiert gleichmässig:] in $\mathbb{D}$ falls die Funktionenfolge $S_n(x) := \sum_{k=0}^{\infty} f_k(x)$ gleichmässig konvergiert.
    \item Für $\mathbb{D} \subset \R$, Folge stetiger Funktionen $f_n:\mathbb{D} \to \R$. Falls $\left| f_n(x) \right| \le c_n \ \forall x \in D$ und $\sum_{n=0}^{\infty} c_n$ konvergent $\implies \sum_{n=0}^{\infty} f_n(x)$ gleichmässig konvergent in $\mathbb{D}$ und Grenzwert $f(x) := \sum_{n=0}^{\infty} f_n(x)$ ist in $\mathbb{D}$ stetig.
\end{compactdesc}

\subsubsection{Potenzreihe}
\begin{compactitem}
    \item $\sum_{k=0}^{\infty} c_kx^k$
\end{compactitem}
\begin{compactdesc}
\item[Posiviten Konvergenzradius:] $\rho$ hat Potenzreihe falls $\limsup_{k \to \infty} \sqrt[^k]{\left| c_k \right| }$ existiert.
    \begin{compactitem}
        \item $\rho = \begin{cases}
            + \infty & \text{if } \limsup_{k \to \infty} \sqrt[^k]{\left| c_k \right| } = 0\\
            \frac{1}{\limsup_{k \to \infty}\sqrt[^k]{\left| c_k \right| }} & \text{if } \limsup_{k \to \infty} \sqrt[^k]{\left| c_k \right| } > 0
        \end{cases}$
    \end{compactitem}
\end{compactdesc}
\begin{compactitem}
    \item Für Potenzreihe mit positiven Konvergenzradius $\rho > 0$ und $f(x):= \sum_{n=1}^{\infty} c_kx^k, |x| < \rho \implies \forall 0 \le r < \rho$ konvergiert $\sum_{k=0}^{\infty} c_kx^k$ gleichmässig auf $[-r, r]$ und $f:]-\rho, \rho[ \to \R$ ist stetig.
    \item Sind stetig im Innern ihres Konvergenzbereiche
\end{compactitem}

\subsection{Trigonometrische Funktionen}
\begin{compactitem}
    \item $\sin(z) = z - \frac{z^2}{3!} + \frac{z^{5}}{5!} - \dots = \sum_{n=0}^{\infty} \frac{(-1)^n z^{2n + 1}}{(2n + 1)!}$.
        \begin{compactitem}
            \item $\sin: \R \to \R$ stetig.
        \end{compactitem}
    \item $\cos(z) = 1 - \frac{z^2}{2!} + \frac{z^{4}}{4!} - \dots = \sum_{n=0}^{\infty} \frac{(-1)^nz^{2n}}{(2n)!}$.
        \begin{compactitem}
            \item $\cos: \R \to \R$ stetig.
        \end{compactitem}
\end{compactitem}
\begin{compactenum}
    \item $\exp(iz) = \cos(x) + i \sin(z) \ \forall z \in \C$
    \item
        \begin{inparaitem}
            \item $\cos(z) = \cos(-z)$
            \item $\sin(-z) = - \sin(z) \ \forall z \in \C$
        \end{inparaitem}
    \item
        \begin{inparaitem}
            \item $\sin(z) = \frac{e^{iz} - e^{-iz}}{2i}$
            \item $\cos(z) = \frac{e^{iz} + e^{-iz}}{2}$
        \end{inparaitem}
    \item
        \begin{inparaitem}
            \item $\sin(z + w) = \sin(z) \cos(w) + \cos(z) \sin(w)$
            \item $\cos(z + w) = \cos(z) \cos(w) - \sin(z) \sin(w)$
        \end{inparaitem}
    \item $\cos(z)^2 + \sin(z)^2 = 1 \ \forall z \in \C$
    \item
        \begin{inparaitem}
            \item $\sin(2z) = 2 \sin(z) \cos(z)$
            \item $\cos(2z) = \cos(z)^2 - \sin(z)^2$
        \end{inparaitem}
\end{compactenum}

\subsection{Kreiszahl}
\begin{compactitem}
    \item $\sin(0) = 0$
    \item $\sin$ hat auf $]0, +\infty[$ min. eine Nullstelle.
    \item für $\pi := \inf \{t > 0 | \sin(t) = 0\} \implies$
        \begin{compactenum}
            \item $\sin(\pi) = 0 \ \pi \in ]2, 4[$
            \item $\forall x \in ]0, \pi[ : \sin(x) > 0$
            \item $e^{\frac{i\pi}{2}} = i$
        \end{compactenum}
    \item $x \ge \sin(x) \ge x - \frac{x^3}{3!} \ \forall 0 \le x \le \sqrt{6}$
\end{compactitem}

$\forall x \in \R$
\begin{compactenum}
    \item
        \begin{inparaitem}
            \item $e^{i \pi} = -1$
            \item $e^{2 \pi i} = 1$
        \end{inparaitem}
    \item
        \begin{inparaitem}
            \item $\sin(x + \frac{\pi}{2}) = \cos(x)$
            \item $\cos(x + \frac{\pi}{2}) = - \sin(x)$
        \end{inparaitem}
    \item
        \begin{inparaitem}
            \item $\sin(x + \pi) = - \sin(x)$
            \item $\cos(x + \pi) = - \cos(x)$
        \end{inparaitem}
    \item
        \begin{inparaitem}
            \item $\sin(x + 2 \pi) = \sin(x)$
            \item $\cos(x + 2 \pi) = \cos(x)$
        \end{inparaitem}
    \item Nullstellen von:
        \begin{compactitem}
            \item $\sin(x) = \{k \cdot \pi | k \in \Z\}$
                \begin{inparaitem}
                    \item $\sin(x) > 0, \ \forall x \in ] 2k \pi, (2k+1) \pi[$ 
                    \item $\sin(x) < 0, \ \forall x \in ](2k + 1) \pi, (2k+2) \pi[$
                \end{inparaitem}
            \item $\cos(x) = \{\frac{\pi}{2} + k \cdot \pi | k \in \Z\}$
                \begin{inparaitem}
                \item $\cos(x) > 0, \ \forall x \in ]\frac{-\pi}{2} + 2k\pi, \frac{-\pi}{2} + (2k+1) \pi[$
                \item $\cos(x) < 0, \ \forall x \in ]\frac{-\pi}{2} + (\frac{2k + 1}{2}\pi, \frac{-\pi}{2} + (2k+2) \pi[$
                \end{inparaitem}
        \end{compactitem}
\end{compactenum}

\begin{compactdesc}
    \item[Tangens:] $\tan(z) := \frac{\sin(z)}{\cos(z)}, z \not\in \{\frac{\pi}{2} + \pi k\}$
    \item[Cotangens:] $\cot(z) := \frac{\cos(z)}{\sin(z)}, z \not\in \{\pi k\}$
\end{compactdesc}


\subsection{Grenzwert von Funktionen}
Function $f: \mathbb{D} \to \R, \mathbb{D} \subset \R$. Grenzwert 
\begin{compactdesc}
    \item[Häufigkeitspunkt:] von $\mathbb{D}$ falls $\forall \delta > 0 (]x_0 - \delta, x_0 + \delta[ \setminus \{x_0\} ) \cap \mathbb{D} \neq \O$.
    \item[Grenzwert:] $A \in \R$ von $f(x)$ für $x \to x_0$ und Funktion $f:\mathbb{D} \to \R$ und Häufigkeitspunkt $x_0 \in \R$ von $\mathbb{D}$. Wird mit $\lim_{n \to \infty} f(x) = A$ bezeichnet falls $\forall \epsilon > 0 \ \exists \delta > 0$ so dass $\forall x \in \mathbb{D} \cap (]x_0 - \delta, x_0 + \delta[ \setminus \{x_0\}) : |f(x) - A| < \epsilon$.
\end{compactdesc}
\begin{compactitem}
    \item $f$ muss am Grenzwert $x_0$ nicht zwingend definiert sein.
    \item Für $f$ und Häufigkeitspunkt $x_0$. $\lim_{n \to \infty}f(x) = A \iff \forall (a_n)_{n \ge 1} \in \mathbb{D} \setminus \{x_0\}$ mit $\lim_{n \to \infty} a_n = x_0$ folgt $\lim_{n \to \infty}f(a_n) = A$.
    \item $f$ ist stetig in $x_0 \in \mathbb{D} \iff \lim_{n \to \infty} f(x) = f(x_0)$.
    \item Für $f,g: \mathbb{D} \to \R$ und falls $\exists \lim_{n \to \infty} f(x)$ und $\lim_{n \to \infty} g(x) \implies$:
        \begin{inparaitem}
            \item $\lim_{n \to \infty} (f + g)(x) = \lim_{n \to \infty} f(x) + \lim_{n \to \infty} g(x)$
            \item $\lim_{n \to \infty} (f \cdot g)(x) = \lim_{n \to \infty} f(x) \cdot \lim_{n \to \infty} g(x)$
        \end{inparaitem}
    \item Für $f,g: \mathbb{D} \to \R, f \le g$ und beide Grenzwerte existieren $\implies \lim_{n \to \infty} f(x) \le \lim_{n \to \infty} g(x)$.
    \item Für $f,g_1, g_2: \mathbb{D} \to \R$, falls $g_1 \le f \le g_2$ und $\lim_{n \to \infty} g_1(x) = \lim_{n \to \infty} g_2(x) \implies \exists \lim_{n \to \infty} f(x)$ und $\lim_{n \to \infty} f(x) = \lim_{n \to \infty} g_1(x)$.
\end{compactitem}

\subsubsection{Links- und rechtsseitige Grenzwerte}
\begin{compactdesc}
    \item[Rechtsseitiger Grenzwert:] $\lim_{x \to x_0^+}$ falls der Grenzwert der eingeschränkten Funktion $f|_{\mathbb{D} \cap [x_0, +\infty[}$ für $x \to x_0$ existiert. Wobei $f:\mathbb{D} \to \R$, $x_0 \in \R$ ein Häufigkeitspunkt von $\mathbb{D} \cap ]x_0, + \infty[$.
    \item[Linksseitiger Grenzwert:] $\lim_{x \to x_0^-}$ falls der Grenzwert der eingeschränkten Funktion $f|_{\mathbb{D} \cap [-\infty, x_0[}$ für $x \to x_0$ existiert. Wobei $f:\mathbb{D} \to \R$, $x_0 \in \R$ ein Häufigkeitspunkt von $\mathbb{D} \cap ]x_0, + \infty[$.
    \item[Erweitert Rechts:] $\lim_{x \to x_0^+} f(x) = +\infty$ falls $\exists \epsilon > 0, \ \exists \delta > 0, \ \forall x \in ]x_0, x_0 + \delta[ \cap : f(x) > \frac{1}{\epsilon}$.
        \begin{compactitem}
        \item Alternativ: $\forall N > 0, \ \exists \delta > 0 \ \forall x \in D \cap ]x_0, x_0 + \delta[ : f(x) > N$.
        \end{compactitem}
    \item[Erweitert Links:] $\lim_{x \to x_0^-} f(x) = -\infty$ falls $\exists \epsilon > 0, \ \exists \delta > 0, \ \forall x \in ]x_0, x_0 + \delta[ \cap : f(x) < -\frac{1}{\epsilon}$.
        \begin{compactitem}
        \item Alternativ: $\forall N > 0, \ \exists \delta > 0 \ \forall x \in D \cap ]x_0, x_0 + \delta[ : f(x) < -N$.
        \end{compactitem}
\end{compactdesc}

\subsubsection{Unendlicher Grenzwert}
\begin{compactdesc}
    \item[Oben:] Für $f: \mathbb{D} \to \R^n, \mathbb{D}$ nach oben beschränkt, so ist $\lim_{x \to +\infty} f(x) = L \in \R$ falls $\forall \epsilon > 0, \ \exists c > 0: \ \forall x \in D, \ x > c \implies |f(x) - L| < \epsilon$
    \item[Unten] Für $f: \mathbb{D} \to \R^n, \mathbb{D}$ nach unten beschränkt, so ist $\lim_{x \to -\infty} f(x) = L \in \R$ falls $\forall \epsilon > 0, \ \exists c > 0: \ \forall x \in D, \ x < -c \implies |f(x) - L| < \epsilon$
\end{compactdesc}
