%! TEX root = ./main.tex

\section{Folgen}

\begin{compactdesc}
    \item[Folge:] $\left( a_n \right)_{n \ge a  > 0}$ ist Funktion $a:\N^* \to \mathbb{A}, n \mapsto a_n, \mathbb{A}$ ist beliebiges Set.
    \item[Konvergent:] ist $(a_n)_{n \in \N}$ falls $\exists l \in R: \quad \forall \epsilon > 0$ das Set $\left\{ n \in \N | a_n \not\in  ] l - \epsilon, l + \epsilon[ \right\} = \left\{ n \in N^* | \left| a_n - l \right| \ge \epsilon  \right\} $ endlich ist
        \begin{compactitem}
            \item $\forall  \epsilon > 0 \ \exists  N \ge 1: \quad |a_n - l| < \epsilon \ \forall n \ge N$
            \item Jede konvergente Folge ist beschränkt
            \item Nicht konvergente Folgen sind divergent
            \item Es gibt $2$ Arten von divergenten Folgen
        \end{compactitem}
    \item[Grenzwert:] $(a_n)_{n \in \N}$ konvergiert gegen $a \iff \lim_{n \to \infty} a_n = a \iff \forall \epsilon > 0, \exists n_0 \in N, \forall n \ge n_0: \left| a_n - a \right| < \epsilon$
        \begin{compactitem}
            \item $\lim_{n \to \infty} a_n = \lim_{n \to \infty} a_{n+k} \forall k \in \N $
        \end{compactitem}
\end{compactdesc}

\subsection{Rechenregeln}
$\forall (a_n)_{n \in \N}, (b_n)_{n \in \N}$ konvergent, mit $\lim_{n \to \infty} a_n = a, \lim_{n \to \infty} b_n = b$
\begin{compactenum}
    \item $(a_n + b_n)_{n \in \N}$ ist konvergent mit $\lim_{n \to \infty} (a_n + b_n) = a + b$
    \item $(a_n \cdot  b_n)_{n \in \N}$ ist konvergent mit $\lim_{n \to \infty} (a_n \cdot  b_n) = a + b$
    \item $\left( \frac{a_n}{b_n} \right) _{n \in \N}$ ist konvergent mit $\lim_{n \to \infty} \left( \frac{a_n}{b_n} \right) = \left( \frac{a}{b} \right)$ if $b_n \neq 0 \ \forall n \in N \wedge  b \neq 0$
    \item Falls $\exists K \ge 1, a_n \le  b_n \ \forall n \ge K \implies a \le b$
\end{compactenum}

\subsection{Monotonie}
\begin{compactdesc}
    \item[Monoton Wachsend:] $a_n \le a_{n+1} \quad \forall n \in \N$
    \item[Strikt Monoton Wachsend:] $a_n < a_{n+1} \quad \forall n \in \N$
    \item[Monoton Fallend:] $a_n \ge a_{n+1} \quad \forall n \in \N$
    \item[Strikt Monoton Fallend:] $a_n > a_{n+1} \quad \forall n \in \N$
\end{compactdesc}

\subsection{Einschliessungskriterium}
    $\lim_{n \to \infty} a_n = \lim_{n \to \infty} b_n = \alpha \in \R \ \exists K \in \N \ \exists \left( c_n \right)_{n \in \N}: a_n \le  c_n \le b_n \ \forall n \ge K \implies \lim_{n \to \infty} c_n = \alpha$

\subsection{Weierstrass / Monoton Konvergenz Satz}
\begin{compactitem}
   \item $(a_n)_{n \in \N}$ monoton wachend und nach oben beschränkt $\implies (a_n)_{n \in \N}$ konvergiert und $\lim_{n \to \infty} a_n = \sup \left\{ a_n : n \ge 1 \right\}$
   \item $(a_n)_{n \in \N}$ monoton fallend und nach unten beschränkt $\implies (a_n)_{n \in \N}$ konvergiert und $\lim_{n \to \infty} a_n = \inf \left\{ a_n : n \ge 1 \right\}$
\end{compactitem} 

\todo{Bekannte Funktionen und deren Grenzwert}

\subsection{Bernoulli Ungleichung}
$\left( 1 + x \right)^{n} \ge  1 + n \cdot x \ \forall n \in N, x > -1 $

\subsection{Limes Superior und Limes Inferior}
Jede beschränkte Folge $(a_n)_{n \in \N}$ kann in zwei monotone Folgen $(b_n)_{n \in \N}$ und $(c_n)_{n \in \N}$ geteilt werden.
\begin{compactenum}
    \item $\forall n \ge 1: b_n = \inf \{a_k | k \ge n\}$ und $c_n = \sup \{a_k | k \ge n\} $
    \item $b_n \le b_{n+1}$ und $c_n \ge c_{n+1} \quad \forall n \in \N$
    \item $(b_n)_{n \in \N}$ monoton wachsend, $(c_n)_{n \in \N}$ monoton fallend
    \item $b_n$ und $c_n$ sind beschränkt $\implies$ konvergent
    \item \textbf{Limes Inferior:} $\liminf_{n \to \infty} a_n := \lim_{n \to \infty} b_n $
    \item \textbf{Limes Superior:} $\limsup_{n \to \infty} a_n := \lim_{n \to \infty} c_n $
    \item $b_n \le  c_n \implies \liminf_{n \to \infty} a_n \le  \limsup_{n \to \infty} a_n $
\end{compactenum}

\begin{compactitem}
    \item $(a_n)_{n \in \N}$ konvergiert $\iff \ (a_n)_{n \in \N}$ beschränkt und $\liminf_{n \to \infty} a_n = \limsup_{n \to \infty} a_n$
\end{compactitem}

\subsection{Cauchy Kriterium}
\begin{compactdesc}
    \item[Cauchy-Folge:] $(a_n)_{n \in \N}$ falls $\forall \epsilon > 0 \ \exists N \in \N: \quad \forall m,n \ge N \ |a_n - a_m| < \epsilon$
        \begin{compactitem}
            \item Abstand zwischen Folgegliedern wird mit wachsendem Index beliebig klein
        \end{compactitem}
\end{compactdesc}

\begin{compactitem}
    \item $a_n$ Cauchy-Folge $\implies a_n$ beschränkt
    \item $a_n$ konvergent $\iff a_n$ Cauchy-Folge
    \item $(a_n)_{n \in \N}$ konvergiert $\iff \ \forall \epsilon > 0 \ \exists \N \ge 1: |a_n - a_m| < \epsilon \quad \forall n,m \ge \N$
    \item $a_n$ nicht Cauchy-Folge $\implies a_n$ divergent
\end{compactitem}

\subsection{Abgeschlossener Teilintervall}
Teilmenge $I \subset \R$
\begin{compactenum}
    \item $[a, b] \quad a \le b, a,b \in \R \implies \text{L}(I) = b - a$
    \item $[a, +\infty[ \quad a \in \R \implies \text{L}(I) = \infty$
    \item $]-\infty, a] \quad a \in \R \implies \text{L}(I) = \infty$
    \item $]-\infty, +\infty[ = \R \implies \text{L}(I) = \infty$
\end{compactenum}

\begin{compactitem}
    \item $I \subset \R$ beschränkt$ \ \iff \text{L}(I) < + \infty$
    \item $I \subset \R$ ist abgeschlossen $\iff $ für jede konvergierende $(a_n)_{n \in \N}$ aus Elementen in $I$ muss $\lim_{n \to \infty} a_n \in I$
    \item $I=[a, b], J=[c, d], a \le b, c \le d, \ a,b,c,d \in \R$, falls $c \le a$ und $b \le d \implies I \subset J$
        \begin{compactitem}
            \item $\text{L}(I) = b - a \le  d - c = \text{L}(J)$
        \end{compactitem}
    \item Monoton fallende Folge von Teilmengen von $\R$ ist eine Folge $(X_n)_{n \in \N}, X_n \subset \R$ mit $X_1 \supseteq X_2, \supseteq \dots \supseteq X_n \supseteq \dots $
\end{compactitem}

\subsection{Cauchy-Cantor}
Für absteigende Folge geschlossener Intervalle $I_1 \supseteq \dots  \supseteq I_n \supseteq \dots $ mit $\text{L}(I_1) < + \infty$ gilt $\bigcap_{n \ge 1} I_n \neq \O $. Falls $\lim_{n \to \infty} \text{L}(I_n) = 0 \implies \bigcap_{n \ge  1} I_n =\{x\} \ x \in \R $.

\subsection{Teilfolge}
Teilfolge von $(a_n)_{n \in \N}$ ist $(b_n)_{n \in \N}$ wobei $b_n = a_{\text{l}(n)}$ für $l:\N^* \to \N^*$ mit der Eigenschaft $\text{l}(n) < l(n + 1) \forall n \ge 1$
\begin{compactitem}
    \item Entsteht durch weglassen von Folgengliedern.
    \item $(a_n)_{n \in \N}$ konvergent $\implies (a_{\text{l}(n)})_{n \in \N}$ konvergent für alle Teilfolgen.
\end{compactitem}

\subsection{Bolzano-Weierstrass}
Jede beschränkte Folge besitzt eine konvergente Teilfolge.
\begin{compactitem}
    \item $(a_n)_{n \in \N}$ beschränkt $\implies$ für jede beschränkte Teilfolge $(b_n)_{n \in \N}$ gilt $\liminf_{n \to \infty} a_n \le \lim_{n \to \infty} b_n \le  \limsup_{n \to \infty} a_n$
    \item Es gibt je eine Teilfolgen von $(a_n)_{n \in \N}$ die $\liminf_{n \to \infty} a_n$ resp. $\limsup_{n \to \infty} a_n$ als Limes annehmen.
\end{compactitem}

\subsection{Folgen in $\R^{d}$ und $\C$}
\todo{Folgen in $\R^d$ und $\C$}

\section{Reihen}
\begin{compactdesc}
    \item[Folge der Partialsummen:] $(S_n)_{n \in \N} := a_1 + a_2 + \dots + a_n = \sum_{k=1}^{n} a_k$ einer Folge $(a_n)_{n \in \N}$.
        \begin{compactitem}
            \item Ist monoton steigend.
        \end{compactitem}
    \item[Reihe:] Unendliche Summe $\sum_{k=1}^{\infty} a_k$ einer Folge $(a_n)_{n \in \N}$.
        \begin{compactitem}
            \item Für divergierende Reihen ist die Summe ein Symbol nicht konvergente Folge $(s_n)_{n \in \N}$.
            \item Für konvergente Reihe ist die Summe ein Symbol für den Grenzwert der Folge $(s_n)_{n \in \N}$.
        \end{compactitem}
    \item[Konvergent:] ist $\sum_{k=1}^{\infty} a_k$ falls die Folge $(S_n)_{n \in \N}$ von $(a_n)_{n \in \N}$ konvergiert.
        \begin{compactitem}
            \item $\sum_{k=1}^{\infty} a_k$ konvergiert $\implies \lim_{k \to \infty} a_k = 0 $.
            \item $\lim_{k \to \infty} a_k \neq  0 \implies \sum_{k=1}^{\infty} a_k$ divergent.
        \end{compactitem}
    \item[Grenzwert:] $\sum_{k=1}^{\infty} a_k := \lim_{n \to \infty} S_n$.
    \item Weglassen von Anfangsgliedern verändert die Konvergenz nicht, verändert ggf. jedoch den Grenzwert.
\end{compactdesc}

\subsection{Bekannte Reihen}
\begin{compactdesc}
    \item[Geometrische Reihe:] $\sum_{k=0}^{\infty} q^k = 1 + q + q^2 + \dots = \frac{1}{1-q} =  
        \begin{cases}
            \text{divergent} & \text{ if } |q| \ge 1\\
            \text{konvergent} & \text{ if } |g| < 1
        \end{cases}
    \ , q \in \C
        $
    \item[Harmonische Reihe:] $\sum_{k=1}^{\infty} \frac{1}{n} = \infty$
    \item[Alternierende Harmon. Reihe:] $\sum_{k=1}^{\infty} \frac{(-1)^{k+1}}{k} = \ln 2$
    \item[Teleskopreihe:] $\sum_{k=1}^{\infty} \frac{1}{k(k+1)} = 1$
\end{compactdesc}

\subsection{Rechenregeln}
$\forall \sum_{k=1}^{\infty} a_k, \sum_{k=1}^{\infty} b_k$ konvergent, $\alpha \in \C$ 
\begin{compactenum}
    \item $\sum_{k=1}^{\infty} (a_k + b_k) = \left( \sum_{k=1}^{\infty} a_k \right) + \left( \sum_{k=1}^{\infty} b_k \right)$ konvergent.
    \item $\sum_{k=1}^{\infty} \alpha \cdot a_k = \alpha \cdot \sum_{k=1}^{\infty} a_k$ konvergent.
\end{compactenum}

\subsection{Cauchy Kriterium}
$\sum_{k=1}^{\infty} a_k$ konvergent $\iff \forall \epsilon > 0 \ \exists N \ge 1: \ \left| \sum_{k=n}^{m} a_k \right| < \epsilon \ \forall m \ge  n \ge N$.
\begin{compactitem}
    \item $\sum_{k=1}^{\infty} a_k$ konvergent $\iff \lim_{k \to \infty} \left| \sum_{k=n}^{m} a_k \right| = 0 \ m \ge n$.
    \item $\sum_{k=1}^{\infty} a_k$ konvergent $\implies \lim_{k \to \infty} a_k = 0 $.
    \item $\lim_{k \to \infty} a_k \neq  0 \implies \sum_{k=1}^{\infty} a_k$ divergent.
\end{compactitem}

\subsection{Anwendung Weierstrass}
$\sum_{k=1}^{\infty} a_k, a_k \ge 0 \ \forall k \in \N^*$ konvergiert $\iff (S_n)_{n \in \N}$ nach oben beschränkt.

\subsection{Vergleichssatz}
Für $\sum_{k=1}^{\infty} a_k$ $\sum_{k=1}^{\infty} b_k, 0 \le a_k \le b_k \ \forall k \ge K$:
\begin{compactdesc}
    \item[Majoranten Kriterium:] $\sum_{k=1}^{\infty} b_k$ konvergiert $\implies \sum_{k=1}^{\infty} a_k$ konvergiert.
    \item[Minoranten Kriterium:] $\sum_{k=1}^{\infty} k_k$ divergiert $\implies \sum_{k=1}^{\infty} b_k$ divergiert.
\end{compactdesc}

\subsection{Absolute Konvergenz}
$\sum_{k=1}^{\infty} a_k$ absolut konvergent falls $\sum_{k=1}^{\infty} |a_k|$ konvergent. 
\begin{compactitem}
    \item Falls divergent, kann sie nur gegen $+\infty$ divergieren.
    \item $\sum_{k=1}^{\infty}| a_k|$ convergent $\implies \sum_{k=1}^{\infty} a_k$ konvergent
        \begin{compactitem}
            \item $\left| \sum_{k=1}^{\infty} a_k \right| \le \sum_{k=1}^{\infty} \left| a_k \right|$.
        \end{compactitem}
\end{compactitem}

\subsection{Leibniz}
$(a_n)_{n \in \N}, a_n \ge 0 \ \forall n \in \N$ monoton fallend und $\lim_{n \to \infty} a_n = 0 \implies S:= \sum_{k=1}^{\infty} \left( -1 \right)^{k+1} a_k$ konvergiert.
\begin{compactitem}
    \item $a_1 - a_2 \le S \le a_1$
    \item $(s_n)_{n \in \N}$ beschränkt, $\lim_{n \to \infty} s_{2n} = \lim_{n \to \infty} s_{2n + 1} = s \implies \lim_{n \to \infty} s_n = s$.
\end{compactitem}
